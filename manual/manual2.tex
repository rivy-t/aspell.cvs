%% LyX 1.1 created this file.  For more info, see http://www.lyx.org/.
%% Do not edit unless you really know what you are doing.
\documentclass[11pt,oneside,english]{book}
\usepackage[T1]{fontenc}
\usepackage[latin1]{inputenc}
\usepackage{geometry}
\geometry{verbose,letterpaper,tmargin=1in,bmargin=1in,lmargin=1in,rmargin=1in}
\usepackage{babel}
\setcounter{secnumdepth}{3}
\setlength\parskip{\medskipamount}
\setlength\parindent{0pt}
\IfFileExists{url.sty}{\usepackage{url}}
                      {\newcommand{\url}{\texttt}}

\makeatletter

%%%%%%%%%%%%%%%%%%%%%%%%%%%%%% LyX specific LaTeX commands.
\providecommand{\LyX}{L\kern-.1667em\lower.25em\hbox{Y}\kern-.125emX\@}

%%%%%%%%%%%%%%%%%%%%%%%%%%%%%% Textclass specific LaTeX commands.
 \newenvironment{lyxcode}
   {\begin{list}{}{
     \setlength{\rightmargin}{\leftmargin}
     \raggedright
     \setlength{\itemsep}{0pt}
     \setlength{\parsep}{0pt}
     \normalfont\ttfamily}%
    \item[]}
   {\end{list}}

%%%%%%%%%%%%%%%%%%%%%%%%%%%%%% User specified LaTeX commands.

\usepackage[T1]{fontenc}
\usepackage{xspace}
\newcommand{\nach}{$\to$\xspace}
\newcommand{\hoch}{\texttt{$^\wedge$}}

\usepackage{html}

\newcommand{\doubledash}{-\hspace{0.1em}-}
\newcommand{\doubledashb}{-\/-}
\newcommand{\dlt}{{\footnotesize$\ll$}}
\newcommand{\dgt}{{\footnotesize$\gg$}}

\begin{htmlonly}

\renewenvironment{lyxcode}
  {\begin{list}{}{
    \setlength{\rightmargin}{\leftmargin}
    \raggedright
    \setlength{\itemsep}{0pt}
    \setlength{\parsep}{0pt}
    \ttfamily}%
   \item[] 
   \begin{ttfamily}}
   {\end{ttfamily}
    \end{list} }

\newenvironment{LyXParagraphIndent}[1]%
{\begin{quote}}
{\end{quote}}

\renewcommand{\LyX}{LyX}

\renewcommand{\doubledash}{\rawhtml &#45;&#45;\endrawhtml}
\renewcommand{\doubledashb}{\rawhtml &#45;&#45;\endrawhtml}
\renewcommand{\dlt}{�}
\renewcommand{\dgt}{�}

\renewcommand{\nach}{\rawhtml <i>to</i> \endrawhtml}
\renewcommand{\hoch}{\rawhtml &#94;\endrawhtml}

\end{htmlonly}

\makeatother
\begin{document}

\title{GNU Aspell 0.50}


\author{Copyright (c) 2000-2002\\
Kevin Atkinson\\
kevin at atkinson dhs org}

\maketitle
\tableofcontents{}


\chapter*{Notes}

A lot has changed since the last version of Aspell (.33.7). This manual
is in the process of being updated, until the update is complete a
lot of this manual may be incorrect. Please keep this in mind. The
latest version of this manual can be found at \url{http://savanah.gnu.org/download/aspell/manual/user/manual.html}

The eventual goal is to convert this manual into Texinfo. However,
since I do not have the time to learn Texinfo right now, I will continue
to maintain it in the present format. Once someone goes through the
trouble of converting it into Texinfo I will maintain the Texinfo
version.


\chapter{Introduction}

{[}FIXME{]}

GNU Aspell is a Free and Open Source spell checker designed to eventually
replace Ispell. Its main feature is that it does a much better job
of coming up with possible suggestions than Ispell does. In fact recent
tests shows that it even does better than Microsoft Word 97's spell
checker in some cases. In addition it has both compile time and run
time support for other non-English languages. Aspell also functions
as a library which can be used by other programs.


\section{Comparison to other spell checker engines}

\vspace{0.3cm}
{\centering \begin{tabular}{|r|l|c|c|c|}
\hline 
&
Aspell&
Ispell&
Netscape 4.0&
Microsoft Word 97\\
\hline 
Open Source&
x&
x&
&
\\
\hline 
Suggestion Intelligence&
88-98&
54&
55-70?&
71\\
\hline 
Personal part of Suggestions&
x&
x&
x&
\\
\hline 
Alternate Dictionaries&
x&
x&
?&
?\\
\hline 
International Support&
x&
x&
?&
?\\
\hline
\end{tabular}\par}
\vspace{0.3cm}

The suggestion Intelligence is based on a small test kernel of misspelled/correct
word pairs. Go to \url{http://aspell.net/test} for more info and
how you can help contribute to the test kernel. The current scores
for aspell are 88 in \textbf{fast} mode, 93 in \textbf{normal} mode,
and 98 in \textbf{bad spellers} mode see section \ref{suggestion}
for more information about the various suggestion modes.

If you have any other information you would like to add to this chart
please contact me at kevina at users sourceforge net.


\subsection{Comparison to Ispell}


\subsubsection{Features that only Aspell has}

\begin{itemize}
\item Does a much better job with coming up with suggestions than Ispell
does or for that matter any other spell checker I have seen. If you
know a spell checker that does a better job please let me know.
\item Can learn from users misspellings.
\item Is an actual library that others programs can link to instead of having
to use it through a pipe.
\item Is multiprocess intelligent. When a personal dictionary (or replacement
list) is saved it will now first update the list against the dictionary
on disk in case another process modified it.
\item Can share the memory used in the main word list between processes.
\item Support for detachable dictionaries \textbf{}so that more than one
aspell class can use the same dictionary.
\item Support for multiple personal dictionaries as well as support for
special auxiliary dictionaries.
\item Better support for run-together words.
\item Ability to use multiple dictionaries by simply specifying it on the
command line or in the configuration files.
\end{itemize}

\subsubsection{Things that, currently, only Ispell have}

\begin{itemize}
\item Support for affix compression (However this should change soon once
Kevin Hendricks is integrated)
\item Lower memory footprint
\item Perhaps better support for spell checking (La)\TeX{} files.
\item Support for spell checking Nroff files.
\end{itemize}

\chapter{Getting Started}


\section{Requirements\label{reqs}}

Aspell requires gcc 2.95 (or better) as the C++ compiler. Other C++
compilers should work with some effort. Other C++ compilers for mostly
POSIX compliant (Unix, Linux, BeOS, CygWin) systems should work with
out any major problems provided that the compile can handle all of
the advanced C++ features Aspell uses. C++ compilers for non-Unix
systems might work but it will take some work. Aspell at very least
requires a Unix-like environment (sh, grep, sed, tr, etc...) in order
to build. Aspell also uses a few POSIX functions when necessary. Nevertheless,
Aspell will compile and run using the MinGW version of gcc provided
that the CygWin environment is used to to build it.


\section{Obtaining}

The latest version can always be found at Aspell's home page at \url{http://aspell.net}.


\section{Support}

Support for Aspell can be found on the Aspell mailing lists. Instructions
for joining the various mailing lists (and an archive of them) can
be found off the Aspell home page at \url{http://aspell.net}.


\section{Helping Out}

{[}FIXME{]}


\section{Compiling \& Installing}


\subsection{Generic Install Instructions}

\begin{quote}
./configure --disable-static \&\& make
\end{quote}
{[}FIXME: Static libraries should be disabled by default, add an explanation
on why static libraries should not be used{]}

For additional configure options type ./configure --help. You can
control what C++ compiler is used by setting the environmental variable
CXX before running configure and you can control what flags are passed
to the C++ compile via the environmental variable CXXFLAGS.

Aspell should then compile with out any additional user intervention.
If you run into problems please first check the sections below as
that might solve your problem.

To install the program simply type

\begin{quote}
make install
\end{quote}
After Aspell is installed at least one dictionary needs to be installed.
You can find them at \url{http://www.gnu.org/software/aspell/}. The
\char`\"{}speller-util\char`\"{} program must be in your path in order
for the dictionaries to install correctly.

If you do not have Ispell or the traditional Unix {}``spell'' utility
installed on your system than you should also copy the compatibly
scripts {}``ispell'' and {}``spell'' located in the scripts/ directory
into your binary directory which is usually /usr/local/bin so that
programs that expect ispell or spell command will work correctly.


\subsection{General Problems}

Aspell does not use a released version of GNU Libtool. If you have
a need to modify configure.ac or any of the Makefile.am's you should
install the CVS version of libtool.


\subsection{Curses Notes}

If you are having problems compiling termios.cc than the most likely
reason is due to incompatibilities with the curses implementation
on your system. If this is the case than you can explicitly disable
the curses library with --disable-curses. By doing this you will lose
the nice full screen interface but hopefully you will be able to at
least get Aspell to compile correctly.

If the curses library is installed in a non-standard location than
you can specify the library and include directory with --enable-curses=<\,{}<lib>\,{}>
and --enable-curses-include=<\,{}<dir>\,{}>.
\textbf{Lib} can either be the complete path of the library (for example
{}``/usr/local/curses/libcurses.a''), the name of the library (for
example {}``ncurses'') or a combined location and library in the
form {}``-L<\,{}<lib dir>\,{}> -l<\,{}<lib>\,{}>''
(for example {}``-L/usr/local/ncurses/lib -lncurses''). \textbf{Dir}
is the location of the curses header files (for example {}``/usr/local/ncurses/include''). 


\subsection{Win32 Notes}

Aspell should be able to compile on Win32 platforms using the Win32
version of gcc which can either be the Cgiwin or the Mingw version
of Gcc 2.95 using the Cgiwin development environment. The Mingw version
of Aspell will have slightly less functionality, but none of which
is noticeable to the end user. In order to get the nice full screen
interface with Mingw when spell checking files a curses implementation
that does not require Cygwin is required. The PDCurses (\url{http://www.lightlink.com/hessling/PDCurses/})
implementation is known to work, other implementations may work however
they have not been tested. See the previous section for information
on specifying the location of the curses library and include file.

If Aspell is compiled with --enable-win32-relocatable and the \textbf{bindir}
is set to the same value as \textbf{prefix} (ie not <\,{}<prefix>\,{}>/bin)
then the Aspell directory (what prefix is set to) can be relocated
anywhere provided that none of the data files are moved around within
the Aspell directory.

The default paths for Aspell are designed for a Unix system and not
a Win32 system so you might want to specify different ones when compiling
Aspell. Also if the HOME environmental variable is not set Aspell
will assume it is the current working directly. This may lead to your
personal word lists being saved in unpredictable locations. To solve
this either compile with --enable-win32-relocatable (see above) or
specify the complete path of the personal and replacement word lists
in aspell.conf. If Aspell is compiled with --enable-win32-relocatable
than the personal word lists are saved in the \textbf{prefix} directory
and the name is changed from {}``\texttt{.aspell.<lang>.{*}}'' to''
\texttt{<lang>.{*}}''.


\section{Upgrading from Aspell .33/Pspell .12}

Aspell has undergone an extremely large number of changes since the
previous Aspell/Pspell release. For one thing Pspell has been merged
with Aspell so there in no longer two separate libraries you have
to worry about.

The functionally of the old Aspell utility is now broken up into to
utilities. {}``aspell'' and {}``speller-util''. The \char`\"{}aspell\char`\"{}
program handles the checking of documents while {}``speller-util''
handles the management of dictionaries.

The way dictionaries are handled has also changed. For one thing the
language names have changed to the two letter iso code. This means
that your old personal word lists won't work without some minor changes.
{[}FIXME A Script should be written to automate the change and it
should be documented here.{]}

The Pspell ABI is now part of Asepll except that the name of everything
has changed due to the renaming of Pspell to Aspell. In particuler
please note the following name changes:

\begin{quote}
pspell -> aspell\\
manager -> speller\\
emulation -> enumeration\\
master\_word\_list -> main\_word\_list
\end{quote}
Please also note that the name of the \textbf{language-tag} option
has changed to \textbf{lang}. However, for backwards compatibily the
language-tag option will still work.

However, you should also be able to build applications that require
Pspell with the new Aspell as a backwards compatibility header file
is provided.

The format of the complied dictionaries has almost certenly changed.
So you will need to recompile them for the new Aspell.


\chapter{Basic Usage}

Two seperate utilties are now used to control Aspell, \char`\"{}aspell\char`\"{}
and \char`\"{}speller-util\char`\"{}. The \char`\"{}aspell\char`\"{}
program handles the checking of documents while \char`\"{}speller-util\char`\"{}
handles the management of dictionaries. For a quick reference on either
utility use {}``--help''.


\section{Spellchecking Individual Files\label{check}}

To use Aspell to spellcheck a file use:

\begin{quote}
aspell check {[}<\,{}<options>\,{}>{]}
<\,{}<filename>\,{}>
\end{quote}
at the command line where <\,{}<filename>\,{}>
is the file you want to check and <\,{}<options>\,{}>
is any number of optional options. Some of the more useful ones include:

\begin{description}
\item [--mode=<\,{}<mode>\,{}>]the mode to
use when checking files. The available modes are none, url, email,
sgml, or tex. See section \ref{filter} for more informations on the
various modes.
\item [--dont-backup]don't create a backup file.
\item [--sug-mode=<\,{}<mode>\,{}>]the suggestion
mode to use where mode is one of ultra, fast, normal, or bad-spellers.
See section \ref{suggestion} for more information on these modes.
\item [--master=<\,{}<name>\,{}>]the main
dictionary to use.
\item [--keymapping=<\,{}<name>\,{}>]the keymapping
to use. Either \textbf{aspell} for the default mapping or \textbf{ispell}
to use the same mapping that the ispell utility uses.
\end{description}
Please see Chapter \ref{customizing} for more information on the
available options.

For example to check the file foo.txt:

\begin{quote}
aspell check foo.txt
\end{quote}
and to check the file foo.txt using the bad-spellers suggestion mode
and the large American English dictionary:

\begin{quote}
aspell check --sug-mode=bad-spellers --master=american-lrg foo.txt
\end{quote}
If the \textbf{mode} option is not given then aspell will use the
extension of the file to determine the current mode. If the extension
is .tex, then \TeX{} mode will be uses, if the extension is .html,
.htm, .php, or .sgml it will check the file in sgml mode, otherwise
it will use url mode. These extensions can be changed via the \textbf{tex-extension}
and \textbf{sgml-extension} options respectfully. See chapter \ref{filter}
for more information on the various modes that can be used.

If Aspell was compiled with curses support and the TERM environmental
variable is set to a capable terminal type than Aspell will use a
nice full screen interface. Otherwise it will use a simpler {}``dumb''
terminal interface where the misspelled word is surrounded by two
'{*}'. In either case the interface should be self explanatory.


\section{Using Aspell with other Applications}


\subsection{With Applications that Expect Ispell}

Aspell can be used as a drop in replacement for Ispell for programs
that use Ispell through a pipe such as Emacs and \LyX{}. It can also
be used with programs that use simple call the ispell command and
expect the original file to be overwritten with the corrected version.
It support the basic features of Ispell, however it does not currently
have a Nroff mode so there may be situations in which you still wish
to use Ispell. Nevertheless, I have been using Aspell for Xemacs and
\LyX{} since the middle of September of 1998 with out any problems.

If you do not have Ispell installed on your system and have installed
the Ispell compatibly script than you should not need to do anything
as most applications will that expect Ispell will work as expected
with Aspell via the Ispell compatibility script.

Otherwise, the recommended way to use Aspell as a replacement for
ispell is to change the Ispell command from within the program being
used. If the program uses ispell in pipe mode simple change ispell
to aspell. If the program calls the \textbf{}ispell command to check
the file change {}``ispell'' \textbf{}with {}``aspell check''.

If that is impossible and the program uses ispell through a pipe than
the run-with-aspell script can be used for programs using ispell in
pipe mode. The format of the script is:

\begin{quote}
run-with-aspell <\,{}<command>\,{}>
\end{quote}
where <\,{}<command>\,{}> is the name
of the program with any optional arguments.

The old method of mapping Ispell to Aspell is discouraged because
it can create compatibility problems with programs that actually require
Ispell such as Ispell's own scripts.


\subsection{With Emacs and Xemacs}

The easiest way to use Aspell with Emacs or Xemacs is to add this
line:

\begin{lyxcode}
(setq-default~ispell-program-name~\char`\"{}aspell\char`\"{})
\end{lyxcode}
to the end of your .emacs file.

For some reason version 3.0 of ispell.el (the lisp program that (x)emacs
uses) want to reverse the suggestion list. To fix this add this line:

\begin{lyxcode}
(setq-default~ispell-extra-args~'(\char`\"{}-{}-reverse\char`\"{}))
\end{lyxcode}
after the previous line in your .emacs file and it should solve the
problem. 

Ispell.el, version 3.1 (December 1, 1998) and better, has the list
reversing problem fixed. You can find it at \url{http://www.kdstevens.com/~stevens/ispell-page.html}.


\subsection{With \LyX{}}

Version 1.0 of \LyX{} provides support for Aspell learning for users
mistake feature.

To use aspell with \LyX{} 1.0 either change the \textbf{spell\_command}
option in the lyxrc file or use the run-with-aspell utility.


\subsection{With VIM}

\textit{(The following section was written by \char`\"{}R. Marc\char`\"{},
rmarc at copacetic net.) }

To use aspell in vim you simply need to add the following line to
your .vimrc file:

\begin{lyxcode}
map~\textasciicircum{}T~:w!<CR>:!aspell~check~\%<CR>:e!~\%<CR>
\end{lyxcode}
I use <Ctrl-T> since that's the way you spell check in pico. In order
to add a control character to your .vimrc you must type <Ctrl-v> first.
In this case <Ctrl-v><Ctrl-t>.

A more useful way to use Aspell, IMHO, is in combination with newsbody
(\url{http://www.image.dk/~byrial/newsbody/}) which is how I use
it since vim is my editor for my mailer and my news reader.

\begin{lyxcode}
map~\textasciicircum{}T~\textbackslash{}1\textbackslash{}2<CR>:e!~\%<CR>

map~\textbackslash{}1~:w!<CR>

map~\textbackslash{}2~:!newsbody~-qs~-n~\%~-p~aspell~check~\textbackslash{}\%f<CR>
\end{lyxcode}

\subsection{With Pine}

To use aspell in pine simply change the option \textbf{speller} to 

\begin{quote}
aspell --mode=email check
\end{quote}
To change the speller option go to the main menu. Type \textbf{S}
for \emph{setup}, \textbf{C} for \emph{config}, then W for \emph{where
is}. Type in \textbf{speller} as the word to find. The speller option
should be highlighted now. Hit enter, type in the above line, and
hit enter again. Then type \textbf{E} for \emph{exit setup} and \textbf{Y}
to save the change.

If you have a strong desire to check other peoples comments change
\textbf{speller} to

\begin{quote}
aspell check
\end{quote}
instead which will avoid switching aspell into email mode.


\chapter{Customizing Aspell\label{customizing}}

The behavior of Aspell can be changed by any number of options which
can be specified at either the command line, the environmental variable
ASPELL\_CONF, a personal configuration file, or a global configuration
file. Options specified on the command line override options specified
by the environmental variable. Options specified by the environmental
variable override options specified by either of the configurations
files. Finally options specified by the personal configuration file
override options specified in the global configuration file. Options
specified in the environmental variable ASPELL\_CONF, a personal configuration
file, or a global configuration file will take effect no matter how
Aspell is used which includes being used by other applications.

Aspell has three basic type of options: \textbf{boolean}, \textbf{value},
and \textbf{list}. \textbf{Boolean} options are either enabled or
disabled, \textbf{value} options take a specific value, and \textbf{list}
options can either have entries added or removed from the list.


\section{Specifying Options}


\subsection{At the Command Line}

All options specified at the command line have the following basic
format:

\begin{quote}
--<\,{}<option>\,{}>{[}=<\,{}<value>\,{}>{]}
\end{quote}
where the '=' can be replaced by whitespace.

However some options also have single letter abbreviations of the
form:

\begin{quote}
-<\,{}<letter>\,{}>{[}<\,{}<optional
whitespace>\,{}><\,{}<value>\,{}>{]}
\end{quote}

\subsubsection{Boolean}

To enable a boolean option simply special the option with out any
corresponding value. For example to ignore accents when checking words
use {}``--ignore-accents''. To disable a boolean option prefix the
option name with a {}``dont-''. For example to not ignore accents
when checking words use {}``--dont-ignore-accents''.

If a boolean option has a single letter abbreviation simply give the
letter corresponding to either enabling or disabling the option with
out any corresponding value. For example to consider run-together
words legal use {}``-C'' or to consider them illegal use {}``-B''


\subsubsection{Value}

To specify a value option simply specify the option with its corresponding
value. For example to set the filter mode to Tex use {}``--mode=tex''.

If a value option has a single letter shortcut simply specify the
single letter short cut with its corresponding value. For example
to use a large american dictionary use {}``-d american-lrg''.


\subsubsection{List}

To add a value to the list prefix the option name with a {}``add-''
and then specify the value to add. For example to add the URL filter
use {}``--add-filter url''. To remove a value from a list option
prefix the option name with a {}``rem-'' and then specify the value
to remove. For example to remove the URL filter use {}``--rem-filter
url''. To remove all items from a list prefix the option name with
a {}``rem-all'' without specify any value. For example to remove
all filters use {}``--rem-all-filter''.


\subsection{Via a Configuration File}

Aspell can also accept options via a personal or global configuration
file. The exact files to used are specified by the options \textbf{per-conf}
and \textbf{conf} respectfully but the personal configuration file
is normally {}``.aspell.conf'' located in the HOME directory and
the global one is normally {}``aspell.conf'' which is located in
the etc directory which is normally {}``/usr/etc'' or {}``/usr/local/etc''.
To find out the particular values for your particular system use {}``aspell
dump config''.

Each line of the configuration file has the format:

\begin{lyxcode}
<\,{}<option>\,{}>~{[}<\,{}<value>\,{}>{]}
\end{lyxcode}
There may any number of spaces between the option and the value however
it can only be spaces, ie there is no '=' between the option name
and the value.

Comments may also be included by preceding them with a '\#' as anything
from a {}``\#'' to a newline is ignored. Blank lines are also allowed.

Values set in the personal configuration file override those in the
global file. Options specified at either the command line or via an
environmental variable override those specified by either configuration
file.


\subsubsection{Boolean}

To specify a boolean option simply include the option followed by
a {}``true'' to enable it or a {}``false'' to disable it. For
example to allow run-together words use {}``run-together true''.


\subsubsection{Value}

To specify a value option simply include the option followed by the
corresponding option. For example to set the default language to german
use {}``lang german''.


\subsubsection{List}

To add a value to the list prefix the option name with a {}``add-''
and then specify the value to add. For example to add the URL filter
use {}``add-filter url''. To remove a value from a list option prefix
the option name with a {}``rem-'' and then specify the value to
remove. For example to remove the URL filter use {}``rem-filter url''.
To remove all items from a list prefix the option name with a {}``rem-all''
without specify any value. For example to remove all filters use {}``rem-all-filter''.


\subsection{Via an Environmental Variable}

The environmental variable ASPELL\_CONF may also be used and it overrides
any options set in the configuration file. The format of the string
is exactly the same as the configuration file except that semicolons
( ; ) are used instead of newlines.


\section{The Options\label{options}}

The following is a list of available options broken down by category.
Each entry has the following format:

\begin{quote}\begin{description}
\item [<<option>>[,<<single letter abbreviations>>]] \textit{(<<type>>)} <<description>>
\end{description}\end{quote}

Where single letter options are specified as they would appear at
the command line, ie with the preceding dash. Boolean single letter
options are specified in the following format:

\begin{quote}
-<\,{}<abbreviation to enable>\,{}>|-<\,{}<abbreviation
to disable>\,{}>
\end{quote}
<\,{}<Option>\,{}> is one of the following:
\textbf{boolean}, \textbf{string}, \textbf{file}, \textbf{dir}, \textbf{integer},
or \textbf{list}. \textbf{String}, \textbf{file}, \textbf{dir}, and
\textbf{integer} types are all value options which can only take a
specific type of value.


\subsection{Dictionary Options}

The following options may be used to control which dictionaries to
use and how they behave (see section \ref{dict-opts} for more information):

\begin{description}
\item [master,-d]\textit{(string)} base name of the dictionary to use. If
this option is specifed than Aspell with either use this dictionary
or die.
\item [dict-dir]\textit{(dir)} location of the main word list
\item [lang]\textit{(string)} language to use, it follows the same format
of the LANG environmental variable on most systems. It consists of
the two letter ISO 639 language code and an optional two letter ISO
3166 country code after a dash or underscore. The default value is
based on the value of the LC\_MESSAGES locale.
\item [size]\textit{(string)} the preferred size of the word list
\item [jargon]\textit{(string)} an extra information to distinguish two
different words lists that have the same lang and size.
\item [word-list-path]\textit{(list)} search path for word list information
files
\item [module-search-order]\textit{(list)} list of available modules, modules
that come first on this list have a higher priority. Currently there
is only one speller module.
\item [personal,-p]\textit{(file)} personal word list file name
\item [repl]\textit{(file)} replacements list file name
\item [extra-dicts]\textit{(list)} extra dictionaries to use.
\item [strip-accents]\textit{(boolean)} strip accents from all words in
the dictionary
\end{description}

\subsection{Checker Options}

These options control the beahvior of Aspell when checking documents.

\begin{description}
\item [ignore,-W](\textit{integer}) ignore words <= n chars
\item [ignore-case]\textit{(boolean)} ignore case when checking words
\item [ignore-accents]\textit{(boolean)} ignore accents when checking words
\item [ignore-repl]\textit{(boolean)} ignore commands to store replacement
pairs
\item [save-repl]\textit{(boolean)} save the replacement word list on save
allkeyboard \textit{(file)} the base name of the keyboard definition
file to use (see section \ref{typo})
\item [sug-mode]\textit{(mode)} suggestion mode = ultra | fast | normal
| bad-spellers (see section \ref{suggestion})
\end{description}

\subsection{Filter Options}

These options modify the behavior of the various filters (see section
\ref{filter} for more information):

\begin{description}
\item [filter]\textit{(list)} add or removes a filter
\item [mode]\textit{(string)} sets the filter mode. Mode is one if none,
url, email, sgml, or tex. (The short cut options '-e' may be used
for email, '-H' for Html/Sgml, or '-t' for Tex)
\item [encoding]\textit{(string)} The encoding the input text is in. Valid
values are {}``utf-8'', {}``iso8859-{*}'', {}``koi8-r'', {}``viscii'',
{}``cp1252'', {}``machine unsigned 16'', {}``machine unsigned
32''. However, the aspell utility will cuuretly only function correctly
with 8-bit encodings. I hope to provide utf-8 support in the future.
The two {}``machine unsigned'' encodings are entended to be used
by other programs using the Aspell librarary and it is unlikly the
Aspell utility will ever support these encodings.
\item [add|rem-email-quote]\textit{(list)} email quote characters
\item [email-margin]\textit{(integer)} num chars that can appear before
the quote char
\item [sgml-check]\textit{(list)} sgml attributes to always check.
\item [sgml-extension]\textit{(list)} sgml file extensions.
\item [tex-command]\textit{(list)} \TeX{} commands
\item [tex-check-comments]\textit{(boolean)} check \TeX{} comments
\end{description}

\subsection{Run-together Word Options}

These may be used to control the behavior of run-together words (see
section \ref{run-together} for more information):

\begin{description}
\item [run-together,-C|-B]\textit{(boolean)} consider run-together words
legal
\item [run-together-limit]\textit{(integer)} maximum numbers that can be
strung together 
\item [run-together-min]\textit{(integer)} minimal length of interior words
\end{description}

\subsection{Misc Options}

Misc. other options that don't fall under any other catagory

\begin{description}
\item [conf]\textit{(file)} main configuration file
\item [conf-dir]\textit{(dir)} location of main configuration file
\item [data-dir]\textit{(dir}) location of language data files
\item [local-data-dir]\textit{(dir)} alternative location of language data
files. This directory is searched before \textbf{data-dir}. It defaults
to the same directory the actual main word list is in (which is not
necessarily dict-dir).
\item [home-dir](\textit{dir}) location for personal files
\item [per-conf]\textit{(file)} personal configuration file
\item [prefix]\textit{(dir)} prefix directory
\item [set-prefix]\textit{(boolean)} set the prefix based on executable
location (only works on Win32 and when compiled with --enable-win32-relocatable)
\end{description}

\subsection{Aspell Utility Options}

\begin{description}
\item [backup,-b|-x]\textit{(boolean)} create a backup file by appending
{}``.bak'' to the file name. (Only applies when the command is \textbf{check})
\item [time]\textit{(boolean)} time load time and suggest time in pipe mode.
\item [reverse]\textit{(boolean)} reverse the order of the suggestions list.
\item [keymapping]\textit{(string)} the keymapping to use. Either \textbf{aspell}
for the default mapping or \textbf{ispell} to use the same mapping
that the ispell utility uses.
\end{description}

\section{Dumping Configuration Values}

To find out the current value of all the options use the command {}``aspell
dump config''. This will dump the current configuration to standard
output. The format of the contents dumped is such that it can be used
as either the global or personal configuration file.

To find out the current value of a particular option use {}``aspell
config <\,{}<option>\,{}>''. This will
print out the value of <\,{}<option>\,{}>
to stdout and nothing else.


\section{Notes on various Options}


\subsection{Notes on Various Filters and Filter Modes\label{filter}}

Aspell now has rudimentary filter support. You can either select from
individual filters or chose a filter mode. To select a filter mode
use the \textbf{mode} option. You may chose from \textbf{none}, \textbf{url},
\textbf{email}, \textbf{sgml}, and \textbf{tex}. The default mode
is \textbf{url}. Individual filters can be added with the option \textbf{add-filter}
and remove with the \textbf{rem-filter} option. The currently available
filters are \textbf{url}, \textbf{email}, \textbf{sgml}, \textbf{tex}
as well as a bunch of filters which translate the text from one format
to another. 


\subsubsection{None Mode}

This mode is exactly what it says. It turns off all filters.


\subsubsection{Url Filter/Mode}

The \textbf{url} filter/mode skips over URL's, host names, and email
addresses. Because this filter is almost always useful and rarely
does any harm it is enabled in all modes except \textbf{none}. To
turn it off either select the \textbf{none} mode or use \textbf{rem-filter}
option \emph{after} the desired mode is selected.


\subsubsection{Email Filter/Mode}

The \textbf{email} filter/mode skips over quoted text. It currently
does not support skipping over headers however a future version should.
In the mean time I suggest you use Aspell with Newsbody which can
be found at \url{http://home.worldonline.dk/~byrial/newsbody/}. The
option \textbf{email-skip} controls the number of characters that
can appear before the email quote char, the default is 10. The option
\textbf{add|rem-email-quote} controls the characters that are considered
quote characters, the default is {}``>' and '|'.


\subsubsection{SGML Filter/Mode\label{sgml}}

The \textbf{sgml} filter/mode will skip over sgml commands. It currently
does not handle nested < > unless they are in quotes. It also does
it handle the null end tag (net) minimization feature of sgml such
as 

\begin{quote}
<emphasis/important/
\end{quote}
The option \textbf{add|rem-sgml-check} controls which sgml tags should
always be checked. The default is {}``alt''.

The option \textbf{add|rem-sgml-extension} controls which file extensions
are recognized as sgml/html files. The default is html, htm, php,
and sgml. The extension are not case sensitive so extensions like
.HTM will also be recognized.

{[}FIXME: Add note about sgml character decoding and possable enhance
the code to do so{]}


\subsubsection{\TeX{} Filter/Mode}

The \textbf{tex} (all lowercase) filter/mode skips over \TeX{} commands
and parameters and/or options to certain command. It also skips over
\TeX{} comments by default. The option \textbf{{[}dont-{]}tex-check-comments}
controls whether or not aspel will skip over \TeX{} comments. The
option \textbf{add|rem-tex-command} controls which \TeX{} commands
should have certain parameters and/or options also skipped over. Commands
that are not specified will have all there parameters and/or options
checked. The format for each item is

\begin{quote}
<\,{}<command>\,{}>~~<\,{}<a
list of p,P,o and Os>\,{}>
\end{quote}
The first item is simple the command name. The second item controls
which parameters to skip over. A 'p' skips over a parameter while
a 'P' won't. Similar an 'o' will skip over an optional parameter while
a 'O' won't. The first letter on the list will apply to the first
parameter, the second letter will apply to the second parameter etc.
If there are more parameters than letters Aspell will simply check
them as normal. For example the option

\begin{quote}
add-tex-command rule pp
\end{quote}
will skip over the first two parameters of the {}``rule'' command
while the option

\begin{quote}
add-tex-command foo Pop
\end{quote}
will \emph{check} the first parameter of the {}``foo'' command,
skip over the next optional parameter, if it is present, and will
skip over the second parameter --- even if the optional parameter
is not present --- and will check any additional parameters.

A'{*}' at the end of the command is simply ignored. For example the
option

\begin{quote}
enlargethispage p
\end{quote}
will ignore the first parameter in both enlargethispage and enlargethispage{*}.

To remove a command simple use the \textbf{rem-tex-command} option.
For example

\begin{quote}
rem-tex-command foo
\end{quote}
will remove the command foo, if present, from the list of \TeX{} commands.


\subsection{Notes on the Prefix Option}

The \textbf{prefix} option is there to allow Aspell to easily be relocated.
Changing \textbf{prefix} will change all directory names relative
to the new prefix that are not explicitly set. For example if \textbf{prefix}
was {}``/usr/local/aspell'' and \textbf{dict-dir} has a default
value of {}``/usr/local/aspell/dict'' than changing \textbf{prefix}
to {}``/opt/aspell'' will also change the default value of \textbf{dict-dir}
to {}``/opt/aspell/dict''. Note that modifying prefix will only
effect the default compiled in values of directories. If a directory
option is explicitly given a value than changing the value of \textbf{prefix}
has no effect on that directory option.


\subsection{Notes on Typo-Analysis and the Keyboard Definition File\label{typo}}

Aspell .33 and better will, in general, give a higher priority to
certain misspelling which are likely to be due to typos such as {}``teh''
instead of {}``the'' or {}``hapoy'' instead of {}``happy''.
However in order to do this well Aspell needs to know the layout of
the keyboard. The keyboard definition file simply identifies keys
that are right next to each other. The file has an extension of .kbd
and each line consists of two letters corresponding to two keys that
are right next to each other. For example the line {}``as'' will
indicate that '\texttt{\textbf{a}}' and '\texttt{\textbf{s}}' are
right next to each other. If {}``as'' is listed as a entry it is
not necessary to list {}``sa'' as an entry as that will be done
automatically. Also by {}``right next to each other'' I mean to
keys that are close enough together that it is easy to type one instead
of the other. On most keyboards this means keys that are to the left
or to the right of each other and not \emph{keys} that are below or
above it.

The default for this option is normally {}``standard''. However
the default can be changed via the language data file. The normal
default, {}``standard'', should work well for most QWERTY like keyboard
layouts. It may need minor adjusting for foreign keyboards. The {}``dvorak''
option can for a Dvorak layout. When creating a keyboard definition
file for a foreign language please keep in mind that Aspell completely
ignores accents when scoring words so that the key '\texttt{\textbf{o}}'
and '\texttt{\textbf{�}}' will appear to be the same key to aspell
even if they are in fact separate keys on your keyboard.


\subsection{Notes on the Different Suggestion Modes\label{suggestion}}

In order to understand what these suggestion modes do, a basic understanding
of how aspell works is required. See section \ref{works} for that.
The suggestion modes are as follows.

\begin{description}
\item [ultra]This method will use the fastest method available to come up
with decent suggestions. This currently means that it will look for
soundslikes within one edit distance apart without doing any typo
analysis. It is slower than Ispell by a factor of 1.5 to 2 when a
single word list is used. It speed is only minor affected by the size
of the word list, if at all, but it is strongly effected by the number
of word lists use. In this mode Aspell gets about 87\% of the words
from my small test kernel of misspelled words. (Go to \url{http://aspell.net/test}
for more info on the test kernel as well as comparisons of this version
of Aspell with previous versions and other spell checkers.)
\item [fast]This method is like ultra except that it also performs typo
analysis unless it is turned off by setting the keyboard to none.
The typo analysis brings words which are likely to be due to typos
to the beginning of the list but slows things down by a factor of
about two. This mode should get around the same number of words that
the ultra method does.
\item [normal]This method looks for soundslikes within two edit distance
apart and perform typo-analysis unless it is turned off. Is is around
10 times slower than fast mode with the english word list but returns
better suggestions. Its speed is directly proportional to the size
of the word list. This mode gets 93\% of the words.
\item [bad-spellers]This method also looks for soundslikes within two edit
distances apart but is more tailored for the bad speller where as
fast or normal are more tailed to strike a good balance between typos
and true misspellings. This mode never performs typo-analysis and
returns a \emph{huge} number of words for the really bad spellers
who can't seam to get the spelling anything close to what it should
be. If the misspelled word looks anything like the correct spelling
it is bound to be found \emph{somewhere} on the list of 100 or more
suggestions. This mode gets 98\% of the words.
\end{description}

\chapter{Working With Dictionaries\label{manage}\label{dict-opts}}


\section{How Aspell Selects an Appropriate Dictionary}

If the \textbf{master} options is set in any fashion (via the command
line, the ASPELL\_CONF environmental variable, or a configuration
file) Aspell will look for a dictionary of that name. If one could
not be found complain.

Otherwise it will use the value of the \textbf{lang} option to search
for an appropriate dictionary. If more than one dictionary is found
for the given language string than it will look for a dictionary with
a matching jargon if the \textbf{jargon} option is set. If it is not
set it will look for a dictionary with no jargon. If after matching
the lang and jargon there is still more than one dictionary available
it will find one with the size closest to the value of the \textbf{size}
option. The default size is 60. If Aspell can not find a dictionary
based on \textbf{lang} option than it will give up and complain.

If the \textbf{lang} option is not explicitly set its value will be
based on the LC\_MESSAGES locale. This locale is generally taken from
the LC\_MESSAGES environmental variable or the LANG environmental
variable if LC\_MESSAGES is not set. However, if Aspell is being used
as a library from within another program which already explicitly
set the locale then it will use the locale of the library rather than
the environmental variables. If Aspell can not determine the language
from the LC\_MESSAGES locale than it will default to {}``en\_US''.


\section{Listing Available Dictionaries}

For a list of available dictionaries use the command {}``aspell dump
dicts''. This will form a list of dictionaries that aspell will search
when a dictionary is not specifically given.


\section{Dumping the contents of the word list}

The dump command in \textbf{speller-util} will simply dump the contents
of a word list to stdout in a format than can be read back in with
\textbf{speller-util create}.

If no word list is specified the command will act on the default one.
For example the command 

\begin{quote}
speller-util dump personal
\end{quote}
will simply dump the contents of the current personal word list to
stdout.


\section{Creating an Individual Word List}

To create an individual main word list from a list of words use the
command

\begin{quote}
speller-util --lang=<\,{}<lang>\,{}>
create master ./<\,{}<base>\,{}> < <\,{}<wordlist>\,{}>
\end{quote}
where <\,{}<base>\,{}> is the name of
the word list and <\,{}<word list>\,{}>
is the list of words separated by white space. The name of the word
list will automatically be converted to all lowercase. The {}``./''
is important because without it aspell will create the word list in
the normal word list directory. If you are trying to create a word
list in a language other than english check the aspell data-dir (usually
/usr/share/aspell, use {}``aspell dump config'' to find out what
it is on your system) to see if a language data file exists for your
language. If not you will need to create one. See chapter \ref{inter}
for more information on using Aspell with other languages.

This will create the file <\,{}<base>\,{}>
in the current directory. To use the new word list copy the file to
the normal word list directory (use {}``aspell config'' to find
out what it is) and use the option --master=<\,{}<base>\,{}>. 

The compiled dictionary file is machine dependent. It is dependent
on endian order, and the page size for the machine because they are
mmaped in. Please do not distribute the compiled dictionaries unless
you are only distributing them for a particular platform such as you
would a binary. That is why is normally installed in {}``lib/aspell''
instead of {}``share/aspell''.

Aspell is now also able to use special {}``multi'' dictionaries.
See section \ref{dict-opts} form more information.

A personal and replacement word list can be created in a similar fashion.

Because Aspell does not support any sort of affix compression like
Ispell does Ispell word lists will not work as is. In order to use
Ispell's word lists simply pipe the word list through ``ispell -e''
to expand the munched word lists.


\subsection{Format of the Replacement Word List}

The replacement word has each replacement pair on its own line in
the following format

\begin{lyxcode}
<\,{}<misspelled~word>\,{}>:~<\,{}<correction>\,{}>
\end{lyxcode}

\section{Using Multi Dictionaries}

As with precious versions of aspell you can specify the main dictionary
to use via the -d or --master option. However as of Aspell .32 you
can now also:

\begin{enumerate}
\item Specify more than word list to use with the \textbf{extra-dicts} option.
\item Optionally have all accents striped form the word lists using \textbf{strip-accents}
option. This is \emph{not} the same thing as the \textbf{ignore-accents}
option. As enabling the \textbf{ignore-accents} would accept both
cafe and caf� (notice the accent on the e), but only enabling strip-accents
would only accent cafe, even if caf� is in the original dictionary.
Specify \textbf{strip-accents} is just like using a word list with
out the accents.
\item Specify special {}``multi'' dictionaries.
\end{enumerate}
A {}``multi'' dictionary is a special file which basically a list
of dictionary files to use. A multi dictionary must end is \textbf{.multi}
and has roughly the same format of a configuration file where the
two valid keys are \textbf{add} and \textbf{strip-accents}. The \textbf{add}
key is used for adding individual word lists, or other {}``multi''
files. The \textbf{strip-accents} key is used to control if accents
are striped from the dictionaries. Unlike the global strip-accent
option this option only effects word lists that came after the option.
For example:

\begin{quote}
strip-accents yes\\
add english\\
strip-accents no\\
add must-accent
\end{quote}
will strip accents from the english word list but not the must-accent
word list. If the global strip-accents option is specified the local
strip-accents options are ignored

\begin{lyxcode}

\end{lyxcode}

\section{Dictionary Naming\label{dictname}}

In order for Aspell to be able to correctly recognize a dictionary
based on the setting of the LANG environmental variable the dictionaries
need to be located located .... and they need to be a {}``multi''
dictionary.

Each dictionary that yoou expact aspell to be able to find needs to
have the following name:

\begin{quote}
<\,{}<language>\,{}>{[}\_<\,{}<region>{]}{[}-<\,{}<jargon>\,{}>{]}{[}-<\,{}<size>\,{}>{]}.multi
\end{quote}
Where <\,{}<language>\,{}> is the two
letter language code, <\,{}<region>\,{}>
is the two letter region code, <\,{}<jargon>\,{}>
is any extra informations to distinguish the word list from other
ones with the same language and spelling, and <\,{}<size>\,{}>
is the size of the dictionary. If no size is specified that the default
size of 60 will be assumed.

For example:

\begin{quote}
en.multi\\
en\_US.multi\\
en-medical.multi\\
en-medical-85.multi\\
en-85.multi\\
de.pwli
\end{quote}

\section{AWLI files}

In order for Aspell to find dictionaries that are located in odd places
or not named according to section \ref{dictname} a AWLI file needs
to be created for the dictionary and located in some place where Aspell
can find it.

Each AWLI has the the following name:

\begin{quote}
<\,{}<language>\,{}>{[}\_<\,{}<region>{]}{[}-<\,{}<jargon>\,{}>{]}{[}-<\,{}<size>\,{}>{]}-<\,{}<module>\,{}>.pwli
\end{quote}
Where the names have the same meaning as in section \ref{dictname}
and <\,{}<module>\,{}> is the speller
module to use, which should be set to {}``default'' for now since
there is only one speller module.

Each AWLI file for an Aspell word list should then contain exactly
one line which contains the full path of the main word list.


\chapter{Writing programs to use Aspell}

There are two main ways to use aspell from within your application.
Through the external C API or though a pipe. The internal Aspell API
can be used directly but that is not recommended as the actual Aspell
API is constantly changing.


\section{Though the C API}

The Aspell library contains two main classes and several helper classes.
The two main classes are AspellConfig and AspellMaster. The AspellConfig
class is used to set initial defaults and to change spell checker
specific options. The AspellManager class does most of the real work.
It is responsible for managing the dictionaries, checking if a word
is in the dictionary, and coming up with suggestions among other things.
There are many helper classes the important ones are AspellWordList,
AspellMutableWordList, Aspell{*}Enumeration. The AspellWordList classes
is used for accessing the suggestion list, as well as the personal
and suggestion word list currently in use. The AspellMutableWordList
is used to manage the personal, and perhaps other, word lists. The
Aspell{*}Enumeration classes are used for iterating through a list.


\subsection{Usage}

To use Aspell your application should include {}``aspell.h''. In
order to insure that all the necessary libraries are linked in libtool
should be used to perform the linking. When using libtool simply linking
with {}``-laspell'' should be all that is necessary. When using
shared libraries you might be able to simply link {}``-laspell'',
but this is not recommended. This version of Aspell uses the CVS version
of libtool however released versions of libtool should also work.

When your application first starts you should get a new configuration
class with the command:

\begin{lyxcode}
AspellConfig~{*}~spell\_config~=~new\_aspell\_config();
\end{lyxcode}
which will create a new AspellConfig class. It is allocated with new
and it is your responsibility to delete it with delete\_aspell\_config.
Once you have the config class you should set some variables. The
most important one is the language variable. To do so use the command:

\begin{lyxcode}
aspell\_config\_replace(spell\_config,~\char`\"{}lang\char`\"{},~\char`\"{}en\_US\char`\"{});
\end{lyxcode}
which will set the default language to use to American English. The
language is expected to be the standard two letter ISO 639 language
code, with an optional two letter ISO 3166 country code after an underscore.
You can set the preferred size via the \textbf{size} option, any extra
info via the \textbf{jargon} option, and the encoding via the \textbf{encoding}
option. Other things you might want to set is the preferred spell
checker to use, the search path for dictionary's, and the like see
section \ref{options} for a list of all available options.

When ever a new document is created a new AspellManager class should
also be created. There should be one speller class per document. To
create a new speller class use the new\_aspell\_speller and then cast
it up using to\_aspell\_speller like so.

\begin{lyxcode}
AspellCanHaveError~{*}~possible\_err~=~new\_aspell\_speller(spell\_config);

AspellManager~{*}~spell\_checker~=~0;

if~(aspell\_error\_number(possible\_err)~!=~0)

~~puts(aspell\_error\_message(possible\_err));

else

~~spell\_checker~=~to\_aspell\_speller(possible\_err);
\end{lyxcode}
which will create a new AspellManager class using the defaults found
in spell\_config. To find out which dictionary is selected the \textbf{lang},
\textbf{size}, and \textbf{jargon} options may be examinded. To find
out the exact name of the dictionary the \textbf{master} option way
be examned as well as the \textbf{master-flags} opntions to see if
any special flags that were passed on to the module. The \textbf{module}
option way also be examed to figure out which speller module was selected,
but since there is only one this option will always be the same.

If for some reason you want to use different defaults simply clone
spell\_config and change the setting like so:

\begin{lyxcode}
AspellConfig~{*}~spell\_config2~=~aspell\_config\_clone(spell\_config);

aspell\_config\_replace(spell\_config2,~\char`\"{}lang\char`\"{},\char`\"{}nl\char`\"{});

possible\_err~=~new\_aspell\_speller(spell\_config2);

delete\_aspell\_config(spell\_config2);
\end{lyxcode}
Once the speller class is created you can use the check method to
see if a word in the document is correct like so:

\begin{lyxcode}
int~correct~=~aspell\_speller\_check(spell\_checker,~<word>,~<size>);
\end{lyxcode}
<word> can is expected to a const char {*} character string. If the
encoding is set to be {}``machine unsigned 16'' or {}``machine
unsigned 32''. <word> is expected to be a cast from either const
u16int {*} or const u32int{*} respectfully. U16int and u32int are
generally unsigned short and unsigned int respectfully. <size> is
the length of the string or -1 if the sting is null terminated. If
the string is a cast from const u16int {*} or const u32int {*} then
size is the amount of space in bytes the string takes up after being
casted to const char {*} and not the true size of the string. Aspell\_speller\_check
will return 0 is it is not found and non-zero otherwise.

If the word is not correct than the suggest method can be used to
come up with likely replacements.

\begin{lyxcode}
AspellWordList~{*}~suggestions~=~aspell\_speller\_suggest(spell\_checker,~<word>,~<size>);

AspellStringEnumeration~{*}~elements~=~aspell\_word\_list\_elements(suggestions);

const~char~{*}~word;

while~(~(word~=~aspell\_string\_enumeration\_next(aspell\_elements)~!=~NULL~)~\{

~~//~add~to~suggestion~list

\}

delete\_aspell\_string\_manag(elements);
\end{lyxcode}
Notice how \texttt{elements} is deleted but \texttt{suggestions} is
not. The value returned by suggestions is only valid to the next call
to suggest. Once a replacement is made the store\_repl method should
be used to communicate the replacement pair back to the spell checker
(see section \ref{replpair} for why). It usage is as follows:

\begin{lyxcode}
aspell\_speller\_store\_repl(spell\_checker,~

~~~~~~~~~~~~~~~~~~~~~~~~~~<misspelled~word>,~<size>,

~~~~~~~~~~~~~~~~~~~~~~~~~~<correctly~spelled~word>,~<size>);
\end{lyxcode}
If the user decided to add the word to the session or personal dictionary
the the word can be be added using the add\_to\_session or add\_to\_personal
methods respectfully like so:

\begin{lyxcode}
aspell\_speller\_add\_to\_session|personal(spell\_checker,~<word>,~<size>);
\end{lyxcode}
It is better to let the spell checker manage these words rather than
doing it your self so that the words have a change of appearing in
the suggestion list.

Finally, when the document is closed the AspellManager class should
be deleted like so.

\begin{lyxcode}
delete\_aspell\_speller(spell\_checker);
\end{lyxcode}

\subsection{API Reference}

Methods that return a boolean result generally return false on error
and true other wise. To find out what went wrong use the error\_number
and error\_message methods. Unless otherwise stated methods that return
a {}``const char {*}'' will return null on error. In general, the
charter string returned is only valid until the next method which
returns a {}``const char {*}'' is called.

For the details of the various classes please see the header files.
In the future I will generate class references using some automated
tool.


\subsection{Examples}

Two simple examples are included in the examples directory. The {}``example-c''
program demenstracts most of the Aspell library functionary and the
{}``list-dicts'' lists the available dictionaries.


\subsection{Notes About Thread Safety}

Read-only Aspell methods and functions should be thread safe as long
as exceptions, new, delete, delete{[}{]}, and STL allocators are thread
safe. To the best of my knowledge gcc and egcs meet these requirements.
It is up to the programmer to make sure multiple threads do not do
thing such as change the dictionaries and add or delete items from
the personal or session dictionaries.


\section{Through A Pipe\label{pipe}}

When given the \textbf{pipe} or \textbf{-a} command aspell goes into
a pipe mode that is compatible with {}``ispell -a''. Aspell also
defines its own set of extensions to ispell pipe mode.


\subsection{Format of the Data Stream\label{data_stream}}

In this mode, Aspell prints a one-line version identification message,
and then begins reading lines of input. For each input line, a single
line is written to the standard output for each word checked for spelling
on the line. If the word was found in the main dictionary, or your
personal dictionary, then the line contains only a '{*}'. 

If the word is not in the dictionary, but there are suggestions, then
the line contains an '\&', a space, the misspelled word, a space,
the number of near misses, the number of characters between the beginning
of the line and the beginning of the misspelled word, a colon, another
space, and a list of the suggestions separated by commas and spaces.

Finally, if the word does not appear in the dictionary, and there
are no suggestions, then the line contains a '\#', a space, the misspelled
word, a space, and the character offset from the beginning of the
line. Each sentence of text input is terminated with an additional
blank line, indicating that ispell has completed processing the input
line. 

These output lines can be summarized as follows: 

\begin{description}
\item [OK:]{*} 
\item [Suggestions:]\& <\,{}<original>\,{}>
<\,{}<count>\,{}> <\,{}<offset>\,{}>:
<\,{}<miss>\,{}>, <\,{}<miss>\,{}>,
... 
\item [None:]\# <\,{}<original>\,{}> <\,{}<offset>\,{}> 
\end{description}
When in the -a mode, Aspell will also accept lines of single words
prefixed with any of '{*}', '\&', '@', '+', '-', '\textasciitilde{}',
'\#', '!', '\%', or '\textasciicircum{}'. A line starting with '{*}'
tells ispell to insert the word into the user's dictionary. A line
starting with '\&' tells ispell to insert an all-lowercase version
of the word into the user's dictionary. A line starting with '@' causes
ispell to accept this word in the future. A line starting with '+',
followed immediately by a valid mode will cause aspell to parse future
input according the syntax of that formatter. A line consisting solely
of a '+' will place ispell in \TeX{}/\LaTeX{} mode (similar to the
-t option) and '-' returns aspell to its default mode (but these commands
are obsolete). A line '\textasciitilde{}', is ignored for ispell compatibility.
A line prefixed with '\#' will cause the personal dictionaries to
be saved. A line prefixed with '!' will turn on terse mode (see below),
and a line prefixed with '\%' will return ispell to normal (non-terse)
mode. Any input following the prefix characters '+', '-', '\#', '!',
'\textasciitilde{}', or '\%' is ignored, as is any input following.
To allow spell-checking of lines beginning with these characters,
a line starting with '\textasciicircum{}' has that character removed
before it is passed to the spell-checking code. It is recommended
that programmatic interfaces prefix every data line with an uparrow
to protect themselves against future changes in Aspell. 

To summarize these: 

\begin{description}
\item [{*}<\,{}<word>\,{}>]Add a word to the
personal dictionary 
\item [\&<\,{}<word>\,{}>]Insert the all-lowercase
version of the word in the personal dictionary
\item [@<\,{}<word>\,{}>]Accept the word,
but leave it out of the dictionary
\item [\#]Save the current personal dictionary
\item [\textasciitilde{}]Ignored for ispell compatibility.
\item [+]Enter \TeX{} mode.
\item [+<\,{}<mode>\,{}>]Enter the mode specified
by <\,{}<mode>\,{}>.
\item [-]Enter the default mode.
\item [!]Enter terse mode 
\item [\%]Exit terse mode
\item [\textasciicircum{}]Spell-check the rest of the line 
\end{description}
In terse mode, Aspell will not print lines beginning with '{*}', which
indicate correct words. This significantly improves running speed
when the driving program is going to ignore correct words anyway.

In addition to the above commands which are designed for Ispell compatibility
Aspell also supports its own extension. All Aspell extensions follow
the following format.

\begin{quote}
\$\$<\,{}<command>\,{}> {[}data{]}
\end{quote}
Where data may or may not be required depending on the particular
command. Aspell currently supports the following command.

\begin{description}
\item [cs~<\,{}<option>\,{}>,<\,{}<value>\,{}>]Change
a configuration option.
\item [cr~<\,{}<option>\,{}>]Prints the value
of a configuration option.
\item [pp]Returns a list of all words in the current personal wordlist.
\item [ps]Returns a list of all words in the current session dictionary.
\item [l]Returns the current language name.
\item [ra~<\,{}<mis>\,{}>,<\,{}<cor>\,{}>]Add
the word pair to the replacement dictionary for latter use. Returns
nothing.
\end{description}
Anything returned is returned on its own line line. All lists returned
have the following format

\begin{quote}
<\,{}<num of items>\,{}>: <\,{}<item1>\,{}>,
<\,{}<item2>\,{}>, <\,{}<etc>\,{}>
\end{quote}
\emph{(Part of the preceding section was directly copied out of the
Ispell manual)}


\section{Notes of Storing Replacement Pairs\label{replpair}}

The store\_repl method and the \$\$?? should be used because Aspell
is able to learn from users misspellings. For example on the first
pass a user misspells beginning as beging so aspell suggests:

\begin{quote}
begging, begin, being, Beijing, bagging, ....
\end{quote}
However the user then tries \char`\"{}begning\char`\"{} and aspell
suggests

\begin{quote}
beginning, beaning, begging, ... 
\end{quote}
so the user selects beginning. However than, latter on in the document
the user misspelles it as begng (NOT beging). Normally aspell will
suggest.

\begin{quote}
began, begging, begin, begun, ....
\end{quote}
However becuase it knows the user mispelled beginning as beging it
will instead suggest:

\begin{quote}
beginning, began, begging, begin, begun ...
\end{quote}
I myself often misspelled beginning (and still do) as something close
to begging and two many times wind up writing sentences such as \char`\"{}begging
with ....\char`\"{}.

Please also note that replacements commands has a memory. Which means
if you first store the replacement pair:

\begin{quote}
sicolagest -> psycolagest
\end{quote}
then store the replacement pair

\begin{quote}
psycolagest -> psychologist
\end{quote}
The replacement pair

\begin{quote}
sicolagest -> psychologist
\end{quote}
will also get stored so that you don't have to worry about it.


\chapter{Adding Support For Other Languages\label{inter}}

{[}FIXME: This section needs updateing{]}

Before you consider adding support for Aspell for your language first
check if it is already done via the Aspell-Dicts project. As of the
release of support for the following languages is now available:

\begin{quotation}
\begin{tabular}{lll}
Breton&
{[}br{]}&
\\
Catalan&
{[}ca{]}&
\\
Czech&
{[}cs{]}&
\\
Danish&
{[}da{]}&
\\
Dutch&
{[}nl{]}&
\\
Esperanto&
{[}eo{]}&
\\
Faroese&
{[}fo{]}&
\\
French&
{[}fr{]}&
(Standard {[}fr\_FR{]} and Swiss French {[}fr\_CH{]} and in three
sizes: small, medium and large)\\
German&
{[}de{]}&
(Standard {[}de\_DE{]} and Swiss German {[}de\_CH{]})\\
Italian&
{[}it{]}&
\\
Norwegian&
{[}no{]}&
\\
Polish&
{[}pl{]}&
\\
Portuguese&
{[}pt{]}&
(Standard {[}pt\_PT{]} and Brazilian {[}pt\_BR{]})\\
Russian&
{[}ru{]}&
\\
Spanish&
{[}es{]}&
\\
Swedish&
{[}sv{]}&
\\
\end{tabular}
\end{quotation}
You can find all of these packages at the aspell home page (\url{http://aspell.sourceforge.net/}).

If your language is not listed above please send me a note and I will
work with you on adding support. \emph{The instructions below still
apply however for this version of Aspell however they may not once
the Aspell is merged into Pspell (see section \ref{future}).}

Adding a language to aspell is fairly straightforward. You need to
at very least create the language data file, and compile a new word
list. You should also create a PWLI file for each of your word lists
so that your new word lists will work correctly with the LANG environmental
variable and with Pspell.

Please note, however, that Aspell international support is not 100\%
done yet. More information on my future planes for international support
in aspell can be found at \url{http://aspell.sourceforge.net/international/}.


\section{The Language Data File}

The basic format of the language data data is the same as it for aspell
configuration file. It is named <\,{}<lang>\,{}>.dat
and is located in the architecture independent data dir for aspell
(option \textbf{data-dir}) which is usually <\,{}<prefix>\,{}>/share/aspell.
Use {}``aspell config'' to find out where it is in your installation.

The language data file has several mandatory fields, and several optional
ones. All fields are case sensitive and should be in all lower case.

The two mandatory fields are \textbf{name} and \textbf{charset}. \textbf{Name}
is the name of the language and should be the same as the file name
(without the .dat). \textbf{Charset} is the charset aspell will expect
the word lists to be formatted in. You may chose from any of the iso-8859-{*}
character sets as well as, koi8-f, koi8-r, and viscii. If your language
can fit in the plain old ASCII character set use iso8859-1. If you
use some other character set for your language other than the ones
listed here drop me a note and I will look into adding support for
it.

The optional fields are \textbf{special, soundslike}, \textbf{keyboard}
and a bunch of options to specify how run-together words are handles.
\textbf{Special} is for non letter characters that can appear in your
language such as the \textbf{'} and \textbf{-}. The format for the
value is a list separated by spaces. Each item of the list has the
following format

\begin{quote}
<\,{}<char>\,{}>~~<\,{}<begin>\,{}><\,{}<middle>\,{}><\,{}<end>\,{}>
\end{quote}
<\,{}<char>\,{}> is the non letter character
in question. <\,{}<begin>\,{}>,<\,{}<middle>\,{}>,<\,{}<end>\,{}>
are either a '-' or a '{*}'. A star for <\,{}<begin>\,{}>
means that the character at the beginning of the word, a '-' means
it can't. The same is true for <\,{}<middle>\,{}>
and <\,{}<end>\,{}>. For example the
entry for the \textbf{'} in english is:

\begin{lyxcode}
'~-{*}-
\end{lyxcode}
To include more than one middle character just list them one after
another on the same line. for example to make both the ' and the -
a middle character use the following line in the language data file:

\begin{lyxcode}
special~'~-{*}-~-~-{*}-
\end{lyxcode}
The \textbf{soundslike} option, \textbf{}if present, should be the
name of the soundslike data for the language. The data is expected
to be in the file <\,{}<name>\,{}>\_phonetic.dat. 

If the name is \textbf{generic} a really generic soundslike algorithm
will be used which consists of striping all the vowels and removing
all accents. I recommend first using the generic algorithm and then,
after aspell is working with the new language, work on the transformation
array.

If the soundslike name is \textbf{none} then no soundslike lookup
table will be used. This will reduce the size of the compiled word
list by around 50\% but at the sacrifice of suggestion quality. If
the soundslike is none than the soundslike for the word will simply
be the word itself in lowercase, will all accents stripped. For languages
with phonetic spelling the difference will not be very noticeable.
However, for languages with non-phonetic spelling there will be a
noticeable difference. The difference you notice will depend on the
quality of the soundslike data file. If you do not notice much of
a difference for a language with non-phonetic spelling that is a good
indication that the soundslike data is not rough enough---or the words
you are trying are not that badly misspelled.

The keyboard option specifies the base name of the keyboard definition
file to use. See section \ref{typo} for more information.

The options to control how run-together words are handled are the
same as the are in the normal configurations files. Please see section
\ref{run-together} for more information.


\section{Compiling the Word List}

Once you have a working language data file installed in the right
place you are ready to compile the main word list. See section \ref{manage}
to find out what to do. This section also includes instructions for
creating the PWLI file.


\section{Phonetic Code}

\textit{(The following section was written by Bj�rn Jacke, bjoern.jacke
at gmx de)}

Aspell is in fact the spell checker that comes up with the best suggestions
if it finds an unknown word. One reason is that it does not just compare
the word with other words in the dictionary (like Ispell does) but
also uses phonetic comparisons with other words.

The new table driven phonetic code is very flexible and setting up
phonetic transformation rules for other languages is not difficult
but there can be a number of stumbling stones --- that's why I wrote
this section.

The main phonetic code is free of any language specific code and should
be powerful enough to allow setting up rules for any language. Anything
which is language specific is kept in a plain text file and can easily
be edited. So it's even possible to write phonetic transformation
rules if you don't have any programming skills. All you need to know
is how words of the language are written and how they are pronounced.


\subsection{Syntax of the transformation array}

In the translation array there are two strings on each line; the first
one is the search string (or switch name) and the second one is the
replacement string (or switch parameter). The line

\begin{lyxcode}
version~~~<\,{}<version>\,{}>
\end{lyxcode}
is also required to appear somewhere in the translation array. The
version string can be anything but it should be changed when ever
the a new version of the translation array is released. This is important
because it will keep Aspell from using a compiled dictionary with
the wrong set of rules. For example if when coming up with suggestion
for {}``hallo'' aspell will use the new rules to come up with the
soundslike say {}``H{*}L{*}'' but if hello is stored in the dictionary
using the old rules as {}``HL'' instead of {}``H{*}L{*}'' aspell
will never be able to come up with hello. So to solve this problem
Aspell checks if the version strings match and abort with an error
if they don't. Thus it is important to update it when ever a new version
of the translation array is releases. This is only a problem with
the main word list as the personal word lists are now stored as simple
word lists with a single header line (ie, no soundslike data).

Each non switch line represents one replacement (transformation) rule.
Words beginning with the same letter must be grouped together; the
order inside this group does not depend on alphabetical issues but
it gives priorities; the higher the rule the higher the priority.
That's why the first rule that matches is applied. In the following
example:

\begin{lyxcode}


GH~~~~~\_~~\\
G~~~~~~K~~\\
~
\end{lyxcode}
{}``GH \nach \_'' has higher priority than {}``G \nach K''. {}``\_''
represents the empty string {}``''. If {}``GH \nach \_'' would
stand after {}``G \nach K'', the second rule would never match because
the algorithm would stop searching for more rules after the first
match. The above rules transform any {}``GH'' to an empty string
(delete them) and transform any other {}``G'' to {}``K''.

At the end of the first string of a line (the search string) there
may optionally stand a number of characters in brackets. One (only
one!) of these characters must fit. It's comparable with the {[} {]}
brackets in regular expressions. The rule {}``DG(EIY) \nach J''
for example would match any {}``DGE'', {}``DGI'' and {}``DGY''
and replace them with {}``J''. This way you can reduce several rules
to one.

Behind the search string there can stand one or more dashes (-). Those
search strings will be matched totally but only the beginning of the
string will be replaced. Furthermore for these rules no follow-up
rule will be searched (what this is will be explained later). The
rule {}``TCH-- \nach \_'' will match any word containing {}``TCH''
(like {}``match'') but will only replace the first character {}``T''
with an empty string. The number of dashes determines how many characters
from the end will not be replaced. After the replacement the search
for transformation rules continues with the not replaced {}``CH''! 

If a {}``<'' is appended to the search string, the search for replacement
rules will continue with the replacement string and not with the next
character of the word. The rule {}``PH< \nach F'' for example would
replace {}``PH'' with {}``F'' and then again start to search for
a replacement rule for {}``F\ldots{}''. If there would also be
rules like {}``FO \nach O'' and {}``F \nach \_'' then words like
{}``PHOXYZ'' would be transformed to {}``OXYZ'' and any occurrences
of {}``PH'' that are not followed by an {}``O'' will be deleted
like {}``PHIXYZ \nach IXYZ''. The second replacement however is
not applied if the priority of this rule is lower than the priority
of the first rule.

Priorities are added to a rule by putting a number between 0 and 9
at the end of the search string, for example {}``ING6 \nach N''.
The higher the number the higher is the priority.

Priorities are especially important for the previously mentioned follow-up
rules. Follow-up rules are searched beginning from the last string
of the first search string. This is a bit complicated but I hope this
example will make it more clear:

\begin{lyxcode}


CHS~~~~~~X~\\
CH~~~~~~~G~\\
~\\
HAU-{}-1~~~H~\\
~\\
SCH~~~~~~SH~\\
~
\end{lyxcode}
In this example {}``CHS' in the word {}``FUCHS'' would be transformed
to {}``X''. If we take the word {}``DURCHSCHNITT'' the things
look a bit different. Here {}``CH'' belongs together and {}``SCH''
belongs together and both are spoken separately. The algorithm however
first finds the string {}``CHS'' which may not be transformed like
in the previous word {}``FUCHS''. At this point the algorithm can
find a follow up rule. It takes the last character of the first matching
rule ({}``CHS'') which is {}``S'' and looks for the next match,
beginning from this character. What it finds is clear: It finds {}``SCH
\nach SH'', which has the same priority (no priority means standard
priority, which is 5). If the priority is the same or higher the follow-up
rule will be applied. Let's take a look at the word {}``SCHAUKEL''.
In this word {}``SCH'' belongs together and may not be torn apart.
After the algorithm has found \char`\"{}SCH \nach SH\char`\"{} it
searches for a follow-up rule for \char`\"{}H\char`\"{}+\char`\"{}AUKEL\char`\"{}.
It finds \char`\"{}HAU--1 \nach H\char`\"{}, but does not apply it
because its priority is lower than the one of the first rule. You
see that this is a very powerful feature but it also can easily lead
to mistakes. If you really don't need this feature you can turn it
off by putting the line

\begin{lyxcode}
followup~~~~~~0~
\end{lyxcode}
at the beginning of the phonetic table file. As mentioned, for rules
containing a `-' no follow-up rules are searched but giving such rules
a priority is not totally senseless because they self can be follow-up
rules and in that case the priority makes sense again. Follow-up rules
of follow-up rules are not searched because this is in fact not needed
very often.

The control character \char`\"{}\hoch\char`\"{} says that the search
string only matches at the beginning of words so that the rule \char`\"{}RH\hoch \nach R\char`\"{}
will only apply to words like \char`\"{}RHESUS\char`\"{} but not \char`\"{}PERHAPS\char`\"{}.
You can append another \char`\"{}\hoch\char`\"{} to the search string.
In that case the algorithm treats the rest of the word totally separately
from first matched string in at beginning. This is useful for prefixes
whose pronunciation does not depend on the rest of the word and vice
versa like \char`\"{}OVER\hoch\hoch\char`\"{} in English for example.

The same way as \char`\"{}\hoch\char`\"{} works does \char`\"{}\$\char`\"{}
only apply on words that end with the search string. \char`\"{}GN\$ \nach N\char`\"{}
only matches on words like \char`\"{}SIGN\char`\"{} but not \char`\"{}SIGNUM\char`\"{}.
If you use \char`\"{}\hoch\char`\"{} and \char`\"{}\$\char`\"{} together,
both of them must fit \char`\"{}ENOUGH\hoch\$ \nach NF\char`\"{} will
only match the word \char`\"{}ENOUGH\char`\"{} and nothing else.

Of course you can combine all of the mentioned control characters
but they must occur in this order: \texttt{< - priority \^{} \$}.
All characters must be written in CAPITAL letters.

If absolutely no rule can be found --- might happen if you use strange
characters for which you don't have any replacement rule --- the next
character will simply be skipped and the search for replacement rules
will continue with the rest of the word.

If you want double letters to be reduced to one you must set up a
rule like \char`\"{}LL- \nach L\char`\"{}. If double letters in the
resulting phonetic word should be allowed, you must place the line

\begin{lyxcode}
collapse\_result~~~~~0~
\end{lyxcode}
at the beginning of your transformation table file; otherwise set
the value to `1'. The English rules for example strip all vowels from
words and so the word \char`\"{}GOGO\char`\"{} would be transformed
to \char`\"{}K\char`\"{} and not to \char`\"{}KK\char`\"{} (as desired)
if \texttt{collapse\_result} is set to 1. That's why the English rules
have \texttt{collapse\_result} set to \texttt{0}.


\subsection{How do I start finally?}

Before you start to write an array of transformation rules, you should
be aware that you have to do some work to make sure that things you
do will result in correct transformation rules.


\subsubsection{Things that come in handy}

First of all you need to have a large word list of the language you
want to make phonetics for. It should contain about as many words
as the dictionary of the spell checker. If you don't have such a list,
you will probably find an Ispell dictionary at \url{http://fmg-www.cs.ucla.edu/geoff/ispell-dictionaries.html}
which will help you. You can then make affix expansion via \texttt{ispell
-e} and then pipe it trough \texttt{\textbackslash{}tr \char`\"{}
\char`\"{} \char`\"{}\textbackslash{}n\char`\"{}} to put one word
on each line. After that you eventually have to convert special characters
like `�' from Ispell's internal representation to latin1 encoding.
\texttt{sed s/e'/�/g} for example would replace all e' with �.

The second is that you know how to use regular expressions and know
how to use \texttt{grep}. You should for example know that 

\begin{lyxcode}
grep~\^{}{[}\^{}aeiou{]}qu{[}io{]}~wordlist~|~less
\end{lyxcode}
will show you all words that begin with any character but a, e, i,
o or u and then continue with `qui' or `quo'. This stuff is important
for example to find out if a phonetic replacement rule you want to
set up is valid for all words which match the expression you want
to replace. Taking a look at the regex(7) man page is a good idea.


\subsubsection{What the phonetic code should do}

Normal text comparison works well as long as the typer misspells a
word because he pressed one key he didn't really want to press. In
this cases mostly one character differs from the original word.

In cases where the writer didn't know about the correct spelling of
the word however the word may have several characters that differ
from the original word but usually the word would still sound like
the original word. Someone might think for example that `tough' is
spelled `taff'. No spell checker without phonetic code will come to
the idea that this might be `tough' but a spell checker who knows
that `taff' would be pronounced like `tough' will make good suggestions
to the user. Another example could be `funetik' and `phonetic'.

From this examples you can see that the phonetic transformation should
not be too fussy and too precise. If you implement a whole phonetic
dictionary as you can find it in books this will not be very useful
because then there could still be many characters differing from the
misspelled and the desired word. What you should do if you implement
the phonetic transformation table is to reduce the number of used
letters to the only really necessary ones.

Characters that sound similar should be reduced to one. In English
language for example `Z' sounds like `S' and that's why the transformation
rule {}``Z \nach S'' is present in the replacement table. `PH' is
spoken like `F' and so we have a {}``PH \nach F'' rule.

If you take a closer look you will even see that vowels sound very
similar in English language: `contradiction', `cuntradiction', `cantradiction'
or `centradiction' in fact sound nearly the same, don't they? Therefore
the English phonetic replacement rules not only reduce all vowels
to one but even remove them all (removing is done by just setting
up no rule for those letters). The phonetic code of `contradiction'
is `KNTRTKXN' and if you try to read this letter-monster loud you
will hear that it still sound a bit like `contradiction'. You also
see that `D' is transformed to `T' because they nearly sound the same.

If you think you have found a regularity you should \emph{always}
take your word list and grep for the corresponding regular expression
you want to make a transformation rule for. An example: If you come
to the idea that all English words ending on `ough' sound like `AF'
at the end because you think of `enough' and `tough'. If you then
grep for the corresponding regular expression by {}``grep -i ough\$
wordlist'' you will see that the rule you wanted to set up is not
correct because the rule doesn't fit to words like `although' or `bough'.
So you have to define your rule more precisely or you have to set
up exceptions if the number of words that differ from the desired
rule is not so big.

Don't forget about follow-up rules which can help in many cases but
which also can lead to many confusions and side effects. It's also
important to write exceptions in front of the more general rules (\char`\"{}GH\char`\"{}
before \char`\"{}G\char`\"{} etc.).

If you think you have set up a number of rules that may produce some
good results try them out! If you run Aspell as \texttt{}{}``aspell
--lang=<\,{}<your language>\,{}> pipe''
you get a prompt at which you can type in words. If you just type
words Aspell checks them and eventually makes suggestions if they
are misspelled. If you type in {}``\$\$Sw <\,{}<word>\,{}>''
you will see the phonetic transformation and you can test out if your
work does what you want. 

Another good way to control if changes you apply to your rules don't
have any evil side effects is to create another list from your word
list which contains not only the word of the word list but also the
corresponding phonetic version of this word on the same line. If you
do this one time before the change and one time after the change you
can make a diff (see \texttt{man diff}) to see what \emph{really}
changed. To do this use the command {}``aspell --lang=<\,{}<your
language>\,{}> soundslike''. In this mode aspell will
output the the original word and then its soundslike separated by
a tab character for each word you give it. If you are interested in
seeing how the algorithm works you can download a set of useful programs
from \url{http://members.xoom.com/maccy/spell/phonet-utils.tar.gz}.
This includes a program that produces a list as mentioned above and
another program which illustrates how the algorithm works. It uses
the same transformation table as Aspell and so it helps a lot during
the process of creating a phonetic transformation table for Aspell.

During your work you should write down your basic ideas so that other
people are able to understand what you did (and you still know about
it after a few weeks). The English table has a huge documentation
appended for example.

Now you can start experimenting with all the things you just read
and perhaps set up a nice phonetic transformation table for your language
to help Aspell to come up with the best correction suggestions ever
seen also for your language. Take a look at the Aspell homepage to
see if there is already a transformation table for your language.
If there is one you might also take a look at it to see if it could
be improved.

If you think that this section helped you or if you think that this
is just a waste of time you can send any feedback to \emph{}\textit{\emph{bjoern.jacke@gmx.de}}\emph{.}


\section{Controlling the Behavior of Run-together Words\label{run-together}}

Aspell has support for either unconditionally accepting run-together
words or only accepting certain words in compound formation.

Support for unconditionally accepting run-together words can either
be turned on in the language data file or as a normal option via the
\textbf{run-together} option. The \textbf{run-together-limit} options
controls the maximum number of words that can be strung together,
the default is normally 255. The \textbf{run-together-min} options
controls the minimal length the individual components of the run together
word can be, the default is normally 3. Both the run-together-limit
and run-together-min option may be specified in both the language
data file or as a normal. The \textbf{run-together-mid} option, which
may only be specified in the language data file, may be used to specify
up to three optional characters that may appear between individual
words.

In order for aspell to conditionally only accept certain words in
compounds those words must be flagged when the compiled word list
is being created. The format for each entry is

\begin{quote}
<\,{}<word>\,{}>:C{[}1{]}{[}2{]}{[}3{]}<\,{}<middle
char>\,{}>
\end{quote}
The 1, 2, and 3 control if the word is allowed to appear in the begging,
middle, or end of the compound, respectfully. More than one position
flag may be specified. If none of them are specified it as assumed
that the word may appear anywhere. The C is optional if 1, 2, or 3
is specified. The <\,{}<middle char>\,{}>
represents an optional character that may appear after the word in
the formation of the compound if the word is not at the end of the
compound. If the letter is lowercase than the character may appear
after the word, if it is in uppercase then that letter must appear
after the compound. Only one letter may be specified and it must also
be in the list of middle letters specified via the \textbf{run-together-mid}
option. \textbf{}The \textbf{run-together-limit} option may also be
used to specify the maximum number of words to string together. 

For example the word list:

\begin{quote}
beg:1\\
mid:2 \\
end:3 \\
any:C \\
never \\
must:CM \\
maybe:Cm
\end{quote}
Means that the word {}``beg'' may only appear at the begging of
a word, the word {}``mid'' at the middle, the word {}``end'' at
the end, and the word {}``any'' any place. The word {}``never''
is never accepted in a compound unless the \textbf{run-together} option
is set. The word {}``must'' may appear anywhere however it must
be followed by an {}``m'', while the word maybe may be followed
by an {}``m''. Given the above word list the following compounds
or legal:

\begin{quote}
begmidend\\
begany\\
mustmend \\
maybeend \\
maybemend
\end{quote}
are all legal, but the following are not:

\begin{quote}
begmid\\
mustend\\
neverany
\end{quote}
Individual words such as {}``beg'' are always accepted.

When the \textbf{run-together} option is not set Aspell will only
accept words that have been flagged in a run-together word. When the
\textbf{run-together} option is set aspell will accept words which
are as least as long as the value specified in the \textbf{run-together-min}
option. If the words length is less than \textbf{run-together-min}
then it will only accept the word if it has been flagged. When the
\textbf{run-together} option is not set the \textbf{run-together-min}
option is ignored all together.

Currently Aspell only supports run-together words when checking if
a word is in the dictionary. When coming up with suggestions Aspell
treats the word as a normal word and does not do anything special.
This means that the suggestions will be virtually meaningless when
the actual word is a run-together. I plan on more intelligently supporting
run-together words when coming up with suggestions in a future version
of Aspell.


\chapter{How Aspell Works\label{works}}

The magic behind my spell checker comes from merging Lawrence Philips
excellent metaphone algorithm and Ispell's near miss strategy which
is inserting a space or hyphen, interchanging two adjacent letters,
changing one letter, deleting a letter, or adding a letter.

The process goes something like this.

\begin{enumerate}
\item Convert the misspelled word to its soundslike equivalent (its metaphone
for English words).
\item Find all words that have a soundslike within one or two edit distances
from the original words soundslike. The edit distance is the total
number of deletions, insertions, exchanges, or adjacent swaps needed
to make one string equivalent to the other. When set to only look
for soundslikes within one edit distance it tries all possible soundslike
combinations and check if each one is in the dictionary. When set
to find all soundslike within two edit distance it scans through the
entire dictionary and quickly scores each soundslike. The scoring
is quick because it will give up if the two soundslikes are more than
two edit distances apart.
\item Find misspelled words that have a correctly spelled replacement by
the same criteria of step number 2 and 3. That is the misspelled word
in the word pair (such as teh -> the) would appear in the suggestions
list as if it was a correct spelling.
\item Score the result list and return the words with the lowest score.
The score is roughly the weighed average of the weighed edit distance
of the word to the misspelled word and the soundslike equivalent of
the two words. The weighted edit distance is like the edit distance
except that the various edits have weights attached to them.
\item Replace the misspelled words that have correctly spelled replacements
with their replacements and remove any duplicates that might arise
because of this.
\end{enumerate}
Please note that the soundslike equivalent is a rough approximation
of how the words sounds. It is not the phoneme of the word by any
means. For more details about exactly how each step is performed please
see the file \textbf{suggest.cc}. For more information on the metaphone
algorithm please see the data file \textbf{english\_phonet.dat}.

\appendix


\chapter{Changelog}


\section*{Changes from 0.33.7.1 to 0.50}

{[}FIXME{]}

Tons :)


\section*{Changes from .33.7 to .33.7.1 (Aug 20, 2001)}

\begin{itemize}
\item Minor manual fixes.
\item Compile fix for Gcc 3.0 and Solaris.
\end{itemize}

\section*{Changes from .33.6.3 to .33.7 (Aug 2, 2001)}

\begin{itemize}
\item Updates to Autoconf 2.50 and switched to the HEAD branch of libtools.
\item Fixed a bug which caused Aspell to crash when typo-analysis is not
used such as when sug-mode is \textbf{fast} or \textbf{bad spellers}.
\item Added support for typo-analysis even when a soundslike is not used.
\item Fixed a bug which causes extended charters to display incorrectly
on some platforms
\item Compile fixes so that it will compile with Gcc 3.0.
\item Compile fixed which should allow Aspell to compile with Egcs 1.1.
I have not been able to actually test it though. Please let me know
at kevina@users.sourceforge.net if you have have tried with Egcs 1.1.
\item Compile and configuration script fixes so that USE\_FILE\_INO will
properly be defined and Aspell will compile correctly when it is defined.
\item More ANSI C++ compliance fixes.
\end{itemize}

\section*{Changes from .33.6.2 to .33.6.3 (June 3, 2001)}

\begin{itemize}
\item Fixed a build problem in the manual/ directory by including manual-text
and manual-html in the distribution.
\end{itemize}

\section*{Changes from .33.6.1 to .33.6.2 (June 3, 2001)}

\begin{itemize}
\item Compile fix so that Aspell will work correctly when not installed
in /usr/local.
\item Avoided regenerating the manual unless configured with enable-maintainer-mode.
\item Added the missing documentation files in the scowl directory.
\end{itemize}

\section*{Changes from .33.6 to .33.6.1 (May 29, 2001)}

\begin{itemize}
\item Fixed a formating problem with the manual involving <\,{}<.
\item Added a note about creating PWLI files.
\item Removed the space after between the -L and the directory name in the
pspell-module/ Makefile which caused problems on some platforms.
\item Added the configure option AM\_MAINTAINER\_MODE to avoid enabling
rules which often causes generated build files to be rebuild with
the wrong version of Libtool by default. I don't know why I didn't
think to do this a long time ago.
\end{itemize}

\section*{Changes from .33.5 to .33.6 (May 18, 2001)}

\begin{itemize}
\item Fixed a minor bug where some words would have random compound tags
attached to them.
\item Fixed a compile problem on many platforms where fileno is defined
as a macro.
\item Updated the description for a few of Aspell's options.
\item Removed the note of Aspell not being able to run when compiled with
the upcoming Gcc 3.0 compiler as things seam to work now.
\item Added a note about Aspell not being able to compile with Egcs 1.1.
\item Added hack to deal with Libtool's interdependencies problem. See bug
\#416981 for Pspell for more info.
\end{itemize}

\section*{Changes from .33 to .33.5 (April 5, 2001)}

\begin{itemize}
\item {}``dump master'' correctly detects which dictionary and language
to use based on the LANG environmental variable.
\item Fixed a problem on Win32 which involves path names that begin with
<\,{}<Drive Letter>\,{}>:.
\item Bug fixes and enhancements so that Aspell can once again run under
MinGW. You can even use the new full screen interface if Aspell is
compiled with PDCurses.
\item Some major modifications to make Aspell more C++ compliant in order
to get Aspell to compile under the upcoming Gcc 3.0 compiler. This
included only using STL features found in the standard version of
C++. (Which means Aspell will no longer require using the SGI version
of the STL) This should also make compiling C++ under non-gcc compilers
a lot simpler. Please not that Aspell still has some problems with
the upcoming Gcc 3.0 compiler (see section \ref{gcc3.0} for more
info).
\item Minor changes to remove some -Wall warnings. 
\item Added a hack to that Aspell will properly compile as a shared library
under Solaris.
\item Added a few import missing words to the English word list.
\end{itemize}

\section*{Changes from .32.6 to .33 (January 28, 2001)}

\begin{itemize}
\item Added a new new curses based interface to replace the dumb terminal
interface everyone has been bitching about.
\item Added the ability to give higher priority to words such as \char`\"{}the\char`\"{}
instead of \char`\"{}teh\char`\"{} which are likely to be due to typos.
\item Reorganized the manual so that it is hopefully easier to follow.
\item Ability to automatically select the best dictionary to used based
on the setting of the LANG environmental variable.
\item Expanded the medium dictionary size to include more words which included
the original words found in ispell and eliminated the large size for
now.
\item Added three special variant add-on dictionaries.
\item Switched to the multi-language branch of the CVS version of libtool.
\end{itemize}

\section*{Changes from .32.5 to .32.6 (Nov 8, 2000)}

\begin{itemize}
\item Fixed a bug that where Aspell will crash when reading in accented
characters on some platforms. This fixes bug \# 112435.
\item Fixed some other bugs so that it will run under Win32 under CygWin.
Unfortunately it still won't run properly under Mingw.
\item Fixed the mmap test in configure so that it won't fail on some platforms
that use munmap(char {*}, int) instead of munmap(void {*}, int).
\item Upgraded to the latest CVS version of libtool which fixed the problem
with using GNU Make under Solaris.
\item Added an option to copy files instead of using symbolic links for
the special {}``multi'' dictionary files.
\end{itemize}

\section*{Changes from .32.1 to .32.5 (August 18, 2000)}

\begin{itemize}
\item Changed my email from kevinatk at home com to kevina at users sourceforge
net please make a note of the new email address.
\item Added an option to control if the personal replacement dictionary
is saved when the save\_all\_wls method is called.
\item Brought back the ability to dump the master word list even in the
case of the special {}``multi'' lists.
\item Added a large number of hacker related words and some other slang
terms to the medium size word list.
\item Added an {}``ispell'' and {}``spell'' compatibility script for
systems which don't have ispell installed. They are located in the
scripts/ directory and are not installed by default.
\item Manual fixes.
\item Added a note on not using GNU Make on Solaris.
\end{itemize}

\section*{Changes from .32 to .32.1 (August 5, 2000)}

\begin{itemize}
\item Minor compile fixes for recent gcc snapshot.
\item Fixed naming of pwli files.
\item Fixed a bug when aspell will crash when used with certain single letter
flags. This bug was most noticeable when used with Emacs.
\item Word list changes, see SCOWL Readme.
\item Other miscellaneous changes.
\end{itemize}

\section*{Changes from .31.1 to .32 (July 23, 2000)}

\begin{itemize}
\item Added support for optionally doing without the soundslike data.
\item Greatly reduced the amount of memory used when creating word lists.
\item Added support for ignoring accents when coming up with suggestions.
\item Added support for local-data-dir which is searched before data-dir.
\item Added support for specifying which words may be used in compounds
and where they may be used.
\item Added support for having more than one main word list as well as a
special {}``multi'' word list files which will allow multiple word
lists to be treated as one.
\item Aspell now uses a completely new word list.
\item The apostrophe (') is no longer considered part of the word when it
as at the end of the word such as in {}`` dogs' ''.
\end{itemize}

\section*{Changes from .31 to .31.1 (June 18, 2000)}

\begin{itemize}
\item Fixed a bug where Aspell would not create a complete dictionary file
on some platforms when the data is 8-bit.
\item Added a workaround so Aspell will work with ispell.el 3.3.
\item Minor compile fixes so it would compile better with the very latest
gcc (CVS Version).
\item Removed note about compiling in Win32 as I was now able to get it
to work.
\end{itemize}

\section*{Changes from .30.1 to .31 (June 11, 2000)}

\begin{itemize}
\item Added support for spell checking run together words.
\item Added an option to produce a list of misspelled words from standard
input.
\item More robust error reporting when reading in language data files.
\item Fixed a bug when that will cause Aspell to crash if the {}``special''
line was not defined in the language data file.
\item Update Pspell Module.
\item Minor bug fixes.
\item Added cross references in {}``The Aspell utility Chapter'' for easier
use.
\end{itemize}

\section*{Changes from .30 to .30.1 (April 29, 2000)}

\begin{itemize}
\item Ported Aspell to Win32 platforms.
\item Portability fixes which may help aspell compile on other platforms.
\item Aspell will no longer fail if for some reason the mmap fails, instead
it will just read the file in as normal and free the memory when done.
\item Minor changes in the format of the main word list as a result of the
changes, the old format should still work in most cases.
\item Fixed a bug when aspell was ignoring the extension of file names such
as .html or .tex when checking files.
\item Fixed a bug when aspell will go into an infinite loop when creating
the main word list from a word list which has duplicates in it.
\item Minor changes to the manual for better clarity.
\end{itemize}

\section*{Changes from .29.1 to .30 (April 2, 2000)}

\begin{itemize}
\item Fixed many of the capitalization bugs found in previous versions of
Aspell.
\item Changed the format of the main word list yet again.
\item Fixed a bug so that {}``aspell check'' will work on the PowerPC.
\item Added ability to change configuration options in the middle of a session.
\item Added words from /usr/dict/words found on most Linux systems as well
as a bunch of commonly used abbreviation to the word list.
\item Fixed a bug when aspell will dump core after reporting certain errors
when compiled with gcc 2.95 or higher. This involved reworked the
Exception heritage to get around a bug in gcc 2.95.
\item Added a few more commands to the list of default commands the \TeX{}
filter knows about.
\item Aspell will now check if a word only contains valid characters before
adding it to any dictionaries. This might mean that you have to manually
delete a few words from your personal word list.
\item Added option to ignore case when checking a document.
\item Adjusted the parameters of the {}``normal'' suggest mode to so that
significantly less far fetched results are returned in cases such
as tomatoe, which went from 100 suggestions down to 32, at the expense
of getting slightly lower results (less than 1\%),
\item Improved the edit distance algorithm for slightly faster results.
\item Removed the \$\$m command in pipe mode, you should now use {}``\$\$cs
mode,<\,{}<mode>\,{}>'' to set the
mode and {}``\$\$cr mode'' to find out the current mode.
\item Reworked parts of Aspell to use Pspell services to avoid duplicating
code.
\item Added a module for the newly released Pspell. It will get installed
with the rest of aspell.
\item Miscellaneous other bug fixes.
\end{itemize}

\section*{Changes from .29 to .29.1 (Feb 18, 2000)}

\begin{itemize}
\item Improved the \TeX{} filter so that it will accept '@' at the begging
of a command name and ignore trailing '{*}'s. It also now has better
defaults for which parameters to skip.
\item Reworked the main dictionary so that it can be memory mapped in. This
decreases startup time and allows multiple aspell processes to use
the same memory for the main word list. This also also made Aspell
64 bit clean so that it should work on an alpha now.
\item Fix so that aspell can compile on platforms that gnu as is not available
for.
\item Fixed issue with flock so it would compile on FreeBSD.
\item Minor changes in the code to make it more C++ compliant although I
am sure there will still be problems when using some other compiler
other than gcc or egcs.
\item Added some comments to the header files to better document a few of
the classes.
\end{itemize}

\section*{Changes from .28.3 to .29 (Feb 6, 2000)}

\begin{itemize}
\item Fixed a bug in the pipe mode with lines that start with {}``\textasciicircum{}\$\$''.
\item Added support for ignoring all words less than or equal to a specified
length
\item New soundslike code based thanks to the contribution of Bj�rn Jacke.
It now gets all of its data from a table making it easier for other
people to add soundslike code for their native language. He also converted
the metaphone algorithm to table form, eliminating the need for the
old metaphone code.
\item Major redesign of the suggestion code for better results.
\item Changed the format of the personal word lists. In most cases it should
be converted automatically.
\item Changed the format of the main word list.
\item Name space cleanup for more consistent naming. I now use name spaces
which means that gcc 2.8.{*} and egcs 1.0.{*} will no longer cut it.
\item Used file locks when reading and saving the personal dictionaries
so that it truly multiprocess safe.
\item Added rudimentary filter support.
\item Reworked the configuration system once again. However, the changes
to the end user who does not directly use my library should be minimal.
\item Rewrote my code that handles parsing command line parameters so that
it no longer uses popt as it was causing to many problems and didn't
integrate well with my new configuration system.
\item Fixed pipe mode so that it will properly ignore lines starting with
'\textasciitilde{}' for better ispell compatibility.
\item Aspell now has a new home page at \url{http://aspell.sourceforge.net/}.
Please make note of the new URL.
\item Miscellaneous manual fixes and clarifications.
\end{itemize}

\section*{Changes from .28.2.1 to .28.3 (Nov 20, 1999)}

\begin{itemize}
\item Fixed a bug that caused aspell to crash when spell checking words
over 60 characters long.
\item Reworked {}``aspell check'' so that

\begin{enumerate}
\item You no longer have to hit enter when making a choice.
\item It will now overwrite the original file instead of creating a new
file. An optional backup can be made by using the -b option.
\end{enumerate}
\item Fixed a few bugs in data.cc.
\end{itemize}

\section*{Changes from .28.2 to .28.2.1 (Aug 25, 1999)}

\begin{itemize}
\item Fixed the version number for the shared library.
\item Fixed a problem with undefined references when linking to the shared
library.
\end{itemize}

\section*{Changes from .28.1 to .28.2 (Aug 25, 1999)}

\begin{itemize}
\item Fixed a bunch of bugs in the language and configuration classes.
\item Minor changed in the code so that it could compile with the new gcc
2.95(.1).
\item Changed the output of {}``dump config'' so that default values are
given the value \char`\"{}<default>\char`\"{}. This means that the
output can be used to created a configuration file.
\item Added notes on using aspell with VIM.
\end{itemize}

\section*{Changes from .28 to .28.1 (July 27, 1999)}

\begin{itemize}
\item Removed some debug output
\item Changed notes on compiling with gcc 2.8.{*} as I managed to get it
to compile on my school account
\item Avoided included \textbf{stdexcept} in \textbf{const\_string.hh} so
that I could get to compile on my schools account with gcc 2.8.1.
\end{itemize}

\section*{Changes from .27.2 to .28 (July 25, 1999)}

\begin{itemize}
\item Provided an iterator for the replacement classes.
\item Added support for dumping and creating and merging the personal and
replacement word lists.
\item Changed the aspell utility command line a bit, it now used popt.
\item Totally reworked aspell configuration system. Now aspell can get configuration
from any of 5 sources: the command line, the environmental variable
ASPELL\_CONF, the personal configuration file, the global configuration
file, and finally the compiled in defaults. 
\item Totally reworked the language class in preparation for my new language
code. See \url{http://aspell.sourceforge.net/international/} for
more information of what I have in store.
\item Added some options to the configure script: --enable-dict-dir=DIR,
--enable-doc-dir=DIR, --enable-debug, and --enable-opt
\item Removed some old header files.
\item Reorganized the directory structure a bit
\item Made the text version of the manual pages slightly easier to read
\item Used the \textbackslash{}url command for urls for better formating
of the printed version.
\end{itemize}

\section*{Changes from .27.1 to .27.2 (Mar 1, 1999)}

\begin{itemize}
\item Fixed a major bug that caused aspell to dump core when used without
any arguments 
\item Fixed another major bug that caused aspell to do nothing when used
in interactive mode.
\item Added an option to exit in Aspell's interactive mode.
\item Removed some old documentation files from the distribution.
\item Minor changes on to the section on using Aspell with egcs.
\item Minor changes to remove -Wall warnings.
\end{itemize}

\section*{Changes from .27 to .27.1 (Feb 24, 1999)}

\begin{itemize}
\item Fixed a minor compile problem.
\item Updated the section on using Aspell with egcs to it it more clear
why the patch is necessary.
\end{itemize}

\section*{Changes from .26.2 to .27 (Feb 22, 1999)}

\begin{itemize}
\item \textbf{Totally reworked the C++ library which means you may need
to change some things in your code.}
\item Added support for detachable and multiple personal dictionaries in
the C++ class library.
\item The C++ class library now throws exceptions.
\item Reworked aspell ability to learn from users misspellings a bit so
that it now has a memory. See section \ref{replpair} for more information.
\item Upgraded autoconf to version 2.13 and automake to version 1.4 for
better portability.
\item Fixed the configuration so the {}``make dist'' will work. For now
on aspell will be distributed with {}``make dist''.
\item Added support to skip over URL's, email addresses and host names.
\item Added support for dumping the master and personal word list. You can
now also merge a personal word list. Type aspell -help for help on
using this feature. 
\item Reorganized the source code.
\item Started using proper version numbers for the shared library.
\item Fixed a bug that caused a spell to crash when adding certain replacement
pairs.
\item Fixed the problem with duplicate lines when exiting pipe mode for
good.
\end{itemize}

\section*{Changed from .26.1 to .26.2 (Jan 3, 1998)}

\begin{itemize}
\item Fixed another compile problem. Hopefully this time it will really
compile OK on other peoples machines.
\end{itemize}

\section*{Changed from .26 to .26.1 (Jan 3, 1998)}

\begin{itemize}
\item Fixed a small compile problem in \textbf{as\_data.cc}.
\end{itemize}

\section*{Changed from .25.1 to .26 (Jan 3, 1999)}

\begin{itemize}
\item Fixed a bug that causes duplicates items to be displayed in the suggestion
list for good. (If it still does it please send be email.)
\item Added the ability for aspell to learn form the users misspellings.
\item Library Interface changes. Still more to come....
\item Is now multiprocess safe. When a personal dictionary (or replacement
list) is saved it will now first update he list against the dictionary
on disk in case another process modified it.
\item Fixed the bug that caused duplicate output when used non interactively
in pipe mode.
\item Dropped support for gcc 2.7.2 as the C++ compiler.
\item Updated the How Aspell Works (\ref{works}) Chapter.
\item Added support for the ASPELL\_DATA\_DIR environmental variable.
\end{itemize}

\section*{Changes from .25 to .25.1 (Dec 10, 1998)}

\begin{itemize}
\item Fixed the version number so that Aspell reports the correct version
number.
\item Changed the note on gcc 2.7.2 compilers to make it clear that only
the C++ compiler can not be gcc 2.7.2, it is ok if the C compiler
is gcc 2.7.2.
\item Updated the TODO list and reorganized it a bit.
\item Fixed the directory so that all the documentation will get installed
in \$\{prefix\}/doc/aspell instead of half of it in \$\{prefix\}/doc/aspell
and half of it in \$\{prefix\}/doc/kspell.
\end{itemize}

\section*{Changes from .24 to .25 (Nov 23, 1998)}

\begin{itemize}
\item Total rework of how the main word list is stored. Start up time decreased
to about 1/3 of what it was in .24 and memory usage decreased to about
2/3. (When used with the provided word list on a Linux system). \textbf{}Also
the \textbf{format and default locations of the main word list data
files changed} in the process and the data \textbf{is now machine
dependent}. The personal word list format, however, stayed the same.
\item Changed the scoring method to produce slightly better results with
words like the vs. teh. And other simpler misspellings where two letters
are swapped.
\item Fixed the very unpredictable behavior of the '{*}', '\&', '@' commands
in the pipe mode.
\item Added documentations for Aspell pipe mode (also known as ispell -a
compatibility mode)
\item Added a bunch of Aspell specific extensions to the pipe mode and documented
them.
\item Documented the \textbf{to\_soundslike} and \textbf{soundslike} methods
for the \textbf{aspell} class.
\item Changed the scoring method to produce better results for words like
{}``fone'' vs {}``phone'' and other words that have a spelling
that doesn't directly relate to how the word sounds by using the phoneme
equivalent of the word in the scoring of it.
\item Added the \textbf{to\_phoneme} and \textbf{have\_phoneme} methods
to the \textbf{SC\_Language} class.
\item Added the \textbf{to\_phoneme} method to the \textbf{aspell} class.
\item Added the framework for being able to learn from the users misspelling.
Right now it just keep a log of replacements.
\item Re did \textbf{stl\_rope-30.diff}. \textbf{}For some reason the version
of patch on my system refused it.
\item Rewrite of the {}``Using as a replacement for Ispell'' section and
added the \textbf{run-with-aspell} utility as a replacement of the
old method of mapping Ispell to Aspell.
\item Fixed a bug that caused duplicate words to appear in the suggestion
list.
\end{itemize}

\section*{Changes from .23 to .24 (Nov 8, 1998)}

\begin{itemize}
\item Fixed my code so that it can once again compile with g++ 2.7.2.
\item Rewrote the How It Works chapter.
\item Rewrote the Requirement section and added noted on compiling with
g++ 2.7.2.
\item Added a To Do chapter.
\item Added a Glossary and References chapter.
\item Other minor documentation improvements.
\item Internal code documentation improvements.
\end{itemize}

\section*{Changes from .22.1 to .23 (Oct 31, 1998)}

\begin{itemize}
\item Minor documentation fixes.
\item Changed the scoring strategy for words with 3 or less letters. This
cut the number of words returned for these roughly in half.
\item Expanded the word list to also include \textbf{american.0} and \textbf{american.1}
from the Ispell distribution. It now includes \textbf{english.0},
\textbf{english.1}, \textbf{american.0} and \textbf{american.1} from
the directory \textbf{languages/english} provided with Ispell 3.1.20.
\item Added a link to the location of the latest Ispell.el in the documentation.
\item Started a C interface and added some rough documentation for it.
\end{itemize}

\section*{Changes from .22 to .22.1 (Oct 27, 1998)}

\begin{itemize}
\item Minor bug fixes. I was deleting arrays with delete rather than delete{[}{]}.
I was suppressed that this had not created a problem.
\item Added a simple test program to test for a memory leak present on some
systems. (Only debian slink at the moment.) See the file memleak-test.cc
for more info.
\end{itemize}

\section*{Changes from .21 to .22 (Oct 26, 1998)}

\begin{itemize}
\item Major redesign or the scoring method. It now uses absolute distances
rather than relative scores for more consistent results. See suggest.cc
for more info.
\item Suggest code rewritten is several places however the core process
stayed the same.
\item The suggest\_ultra method temporally does nothing. It should be working
again by the next release.
\end{itemize}

\section*{Changes from .20 to .21 (Oct 13, 1998)}

\begin{itemize}
\item Added documentation for aspell::Error
\item \textbf{Changed the library name from libspell to libaspell.} It should
never have been libspell in the first place. Sorry for the incompatibility.
\item Added \textbf{as\_error.hh} to the list of files copied to the include
directory so that you can actually use the library outside of the
source dir.
\item Fixed bug that caused a segmentation fault with words where the only
suggestions was inserting a space or hyphen such as in {}``ledgerline''.
\item Added the \textbf{score} method to \textbf{aspell}.
\item Changed the scoring method to deal with word when the user uses \char`\"{}f\char`\"{}
in place of \char`\"{}ph\char`\"{} a lot better.
\end{itemize}

\section*{Changes from .11 to .20 (Oct 10, 1998)}

\begin{itemize}
\item \textbf{Name change.} \textbf{\emph{}}\textbf{Everything that was
Kspell is now Aspell.} Sorry, the name Kspell was already used by
KDE and I didn't want to cause any confusion.
\item Fixed a bug that causes a segmentation fault when the HOME environmental
variable doesn't exist.
\end{itemize}

\section*{Changes from .10 to .11 (Sep 12, 1998)}

\begin{itemize}
\item Overhaul of the SC\_Language class
\item Added documentation for international support
\item Added documentation for the C++ library
\item Other minor bug fixes.
\end{itemize}

\chapter{To Do}

Words in bold indicate how you should refer to the item when discussing
it with me or others.


\section{Good ideas that are worth implementing}

These items all sound like good ideas however I am not sure when I
will get to implementing then if ever. If you are looking for a way
to contribute picking up on one of these ideas would be a great way
to start. They are presented in no particular order.

\begin{itemize}
\item Use Lawrence Philips' new Double Metaphone algorithm. See \url{http://aspell.sourceforge.net/metaphone/}.
\item Come up with a plug-in for \textbf{gEdit} the gnome text editor.
\item Change languages (and thus dictionaries) based on the information
in the actual document.
\item Come up with a \textbf{nroff mode} for spell checking.
\item Come up with a mode that will skip words based on the symbols that
(almost) always surround the word. (\textbf{Word skipping by context})
\item Create two \textbf{server modes} for Aspell. One that uses the \textbf{DICT}
protocol and one that uses \textbf{ispell -a} method of communication
of some arbitrary port.
\item Come up with \textbf{thread safe personal dictionaries}.
\item Use the \textbf{Hidden Markov Model} to base the suggestions on not
only the word itself but on the context around the word.
\item Having a way to \textbf{email} \textbf{the personal dictionary} and/or
replacement list to a particular address either periodical or when
it grows to a certain size.
\item Be able to \textbf{}accept \textbf{words with spaces in them} as many
languages have words such as as a word in a foreign phrases which
only make sense when followed by other words.
\end{itemize}
The following good ideas where found in the ispell WISHES file so
I thought I would pass them on.

\begin{itemize}
\item Ispell should be smart enough to ignore hyphenation signs, such as
the \TeX{} \textbackslash{}- hyphenation indicator.
\item (Jeff Edmonds) The personal dictionary should be able to remove certain
words from the master dictionary, so that obscure words like \char`\"{}wether\char`\"{}
wouldn't mask favorite typos.
\item (Jeff Edmonds) It would be wonderful if ispell could correct inserted
spaces such as \char`\"{}th e\char`\"{} for \char`\"{}the\char`\"{}
or even \char`\"{}can not\char`\"{} for \char`\"{}cannot\char`\"{}.
\item Since ispell has dictionaries available to it, it is conceivable that
it could automatically determine the language of a particular file
by choosing the dictionary that produced the fewest spelling errors
on the first few lines.
\end{itemize}

\section{Notes and Status of various items}


\subsection{General region skipping}

I want to implement this give other people an idea of how it should
be done and because I am really sick of having to spell check through
url and email address.


\subsection{Word skipping by context}

This was posted on the Aspell mailing list on January 1, 1999:

I had an idea on a great general way to determine if a word should
be skipped. Determine the words to skip based on the symbols that
(almost) always surround the word.

For example when asked to check the following C++ code:

\begin{lyxcode}
cout~<\/<~\char`\"{}My~age~is:~\char`\"{}~<\/<~num~<\/<~endl;~~\\
cout~<\/<~\char`\"{}Next~year~I~will~be~\char`\"{}~<\/<~num~+~1~<\/<~endl;~
\end{lyxcode}
cout, num, and endl will all be skipped. \char`\"{}cout\char`\"{}
will be skipped because it is always preceded by a <\/<. \char`\"{}num\char`\"{}
will be skipped because it is always preceded by a <\/<. And \char`\"{}endl\char`\"{}
will be skipped because it is always between a <\/< and a ;.

Given the following html code.

\begin{lyxcode}
<table~width=50\%~cellspacing=0~cellpadding=1>~~\\
<tr><td>One<td>Two<td>Three~~\\
<tr><td>1<td>2<td>3~~\\
</table>~\\
~\\
<table~cellspacing=0~cellpadding=1>~~\\
</table>
\end{lyxcode}
table, width cellspacing, cellpadding, tr, td will all be skipped
because they are always enclosed in \char`\"{}<>\char`\"{}. Now of
course table and width would be marked as correct anyway however there
is no harm in skipping them.

So I was wondering if anyone on this list has any experience in writing
this sort of context recognition code or could give me some pointers
in the right direction.

This sort of word skipping will be very powerful if done right. I
imagine that it could replace specific spell checker modes for Tex,
Nroff, SGML etc because it will automatically be able to figure out
where it should skip words. It could also probably do a very good
job on programming languages code.

If you are interested in helping be out with this or just have general
comments about the idea please let me know.


\subsection{Hidden Markov Model}

Knud Haugaard S�rensen suggested this one. From his email on the Aspell
mailing list:

\begin{quote}
consider this examples.

a fone number. -> a phone number. \\
a fone dress. -> a fine dress.

the example illustrates that the right correction might depend on
the context of the word. So I suggest that you take a look on HMM
to solve this problem.

This might also provide a good base to include grammar correction
in aspell.

see this link \url{http://www.cse.ogi.edu/CSLU/HLTsurvey/ch1node7.html}
\end{quote}
I think it is a great idea. However unfortunately it will probably
be very complicated to implement. Perhaps in the far future.


\subsection{Email the Personal Dictionary}

Some one suggest in a personal email:

\begin{quote}
Have you thought of adding a function to aspell, that - when the personal
dictionary has grown significantly - sends the user's personal dictionary
to the maintainer of the corresponding aspell dictionary? (if the
user allows it)

It would be a very useful service to the dictionary maintainers, and
I think most users can see their benefit in it too.
\end{quote}
And I replied:

\begin{quote}
Yes I have considered something like that but not for the personal
dictionaries but rather the replacement word list in order to get
better test data for \url{http://aspell.sourceforge.net/test/}. The
problem is I don't know of a good way to do this sense Aspell can
also be used as a library. It also is not a real high priority, especially
sense I would first need to learn how to send email within a C++ program.
\end{quote}

\subsection{Words With Spaces in Them}

While this is something I would like to do it is not a simple task.
The basic problem is that when tokenizing a string there is no good
way to keep phrases together. So the solution is to some how add special
conditions to certain words which will dictate which words can come
before/after it. Then there is also a problem of how to come up with
intelligent suggestions. What further complicates things is that many
applications send words to Aspell a word at a time. So even if Aspell
did support such a thing many applications that would use Aspell will
not. So, in order for this to work applications will need to send
text to Aspell a document or at least a sentence at a time. Unfortunately
the framework for doing this is not there yet. It will be once I finish
the filter interface. Another possible is to provide call back functions
in which Aspell will be able to request the previous or next word
on request. Yet again the framework for doing this is not there. Perhaps
sometime in the near future.


\chapter{Support for Gcc 2.7.2}

(and other non-standard compliance compilers)

My original plan was to program in such a way that it would Aspell
would compile under Gcc 2.7.2 however after releasing a rather nasty
bug in 2.7.2 with nested types I have desired to drop all support
for Gcc 2.7.2. As of Aspell .27 all hope for being able to compile
under Gcc 2.7.2 is lost as I am now using many modern C++ features
which are simply not present in Gcc 2.7.2, most notably template specialization
and template members. Egcs 1.1.1 is a very good standards compliant
compiler and that is now the officially supported compiler. However
as Egcs 1.1.1 is relatively new and, except for namespaces, provides
little new functionality I will continue to support Egcs 1.0.3. Gcc
2.8.1 should in theory work however it is so buggy I have yet to get
it to compile with it nor has anyone else that I know of.

Yes, my code could be rewritten so that is could compile under Gcc
2.7.2 but Why? Using modern C++ has probably accelerated the development
of this library my at least 50\%. And for that matter why stop at
Gcc 2.7.2 why not go all out and totally rewrite my code pure C. I
hope you see my point.

However that does not mean I want to sacrifice portability unnecessarily.
If you see any of part of my code that in not Standard C++ please
let me know. My hope is that my code could compile on all Standard
compliance C++ compilers with the addition of a few extra header files
from SGI's STL.

As a side note I think that Mozilla's C++ portability guide (\url{http://www.mozilla.org/docs/tplist/catBuild/portable-cpp.html})
could be summed up in one sentence: Program in the dark ages of C++. 


\chapter{Credits}

\begin{itemize}
\item To the many authors of Ispell (including R. E. Gorin, Pace Willisson,
and Geoff Kuenning) for providing me with a good word list as well
as giving me a few good ideas.
\item Alan Beale for going well beyond the call of duty with helping me
create a better word list.
\item Lawrence Philips for coming up with the original metaphone algorithm
and Michael Kuhn for writing C code for the algorithm.
\item Bj�rn Jacke for coming up with a generic soundslike algorithm which
gets all of its data from a file, thus eliminating almost all need
for language specific code from aspell.
\item To the authors of SGI STL version 3.0 and up for proving a great set
of generic container classes with cut the development time of this
program in half at least.
\item To the \LyX{} development team for giving me a great tool for the
development of this manual.
\end{itemize}

\chapter{Glossary and References}

\begin{description}
\item [affix]in grammar, a word element that, when added to a word, modifies
its meaning or function; prefix, infix, or suffix.
\item [Debian]A 100\% Open Source Linux distribution \url{http://www.debian.org}
\item [DICT~protocol]A TCP transaction based query/response protocol that
allows a client to access dictionary definitions from a set of natural
language dictionary databases. \url{http://www.dict.org/}
\item [Gnome]A project to build a complete, user-friendly desktop based
entirely on free software. \url{http://www.gnome.org/}
\item [GTK+]A library for creating graphical user interfaces for the X Window
System. It is designed to be small, efficient, and flexible. \url{http://www.gtk.org/}
\item [GUI]Graphics User Interface
\item [Ispell.el]Emacs interface for ispell. \url{http://www.kdstevens.com/~stevens/ispell-page.html}
\item [Ispell]An international spell checker which is just about the only
decent Open Source spell checker out there. (except Aspell of course).
\url{http://fmg-www.cs.ucla.edu/geoff/ispell.html}
\item [Ispell~-a]An Ispell mode that is designed to be used by other applications
though a pipe.
\item [KDE]A powerful graphical desktop environment for Unix workstations.
\url{http://www.kde.org}
\item [Linux]A Open Source version of Unix which runs on many platforms.
\url{http://www.linux.org/}
\item [\LyX{}]An What You See is What You Mean document editor for the X
environment, \url{http://www.lyx.org}
\item [Open~Source]Software where the source code is available for anyone
to extend or modify. \url{http://www.opensource.org/}
\item [Red~Hat]A commercial Linux distribution. \url{http://www.redhat.com}
\item [RPM]Red Hat's packing format also used by other Linux distributions.
\item [STL]Standard Temple Library. A C++ library of container classes,
algorithms, and iterators.
\item [SGI~STL]Silicon Graphics implantation of the STL. \url{http://www.sgi.com/Technology/STL/}
\item [STLPort]A port of SGI STL designed to run on compilers that don't
support all the latest features of C++. \url{http://corp.metabyte.com/~fbp/stl/}
\end{description}

\chapter{Copyright}

This manual is Copyright (c) 2002 Kevin Atknson. Permission is granted
to copy, distribute and/or modify this document under the terms of
the GNU Free Documentation License, Version 1.1 or any later version
published by the Free Software Foundation; with no Invariant Sections,
no Front-Cover Texts. and no Back-Cover Texts. A copy of the license
is included in the section .

The library and utility program is Copyrighted (c) 2000-2002 by Kevin
Atkinson. You can redistribute it and/or modify it under the terms
of the GNU Lesser General Public License as (LGPL) published by the
Free Software Foundation; either version 2.1 of the License, or (at
your option) any later version.

Certain parts of the library, as indicated at the top of the source
file, are under a weaker license. However, all parts of the library
or LGPL Compatable.

The two licenses follow:

((FDL))


\section{GNU Lesser Public License}

((LGPL))
\end{document}
